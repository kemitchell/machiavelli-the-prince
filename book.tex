\documentclass[12pt,letterpaper]{memoir}

\title{\textbf{The Prince}}
\author{%
Nicoló Machiavelli\\
Translated into English by Luigi Ricci}
\date{}

\begin{document}
\frontmatter
\maketitle
\tableofcontents
\mainmatter

It is customary for those who wish to gain the favour of a prince to
endeavour to do so by offering him gifts of those things which they
hold most precious, or in which they know him to take especial delight.
In this way princes are often presented with horses, arms, cloth of
gold, gems, and such-like ornaments worthy of their grandeur. In my
desire, however, to offer to Your Highness some humble testimony of
my devotion, I have been unable to find among my possessions anything
which I hold so dear or esteem so highly as that knowledge of the deeds
of great men which I have acquired through a long experience of modern
events and a constant study of the past.

The results of my long observations and reflections are recorded in the
little volume which I now offer to Your Highness: and although I deem
this work unworthy of Your Highness's notice, yet my confidence in your
humanity assures me that you will accept it, knowing that it is not
in my power to offer you a greater gift than that of enabling you to
understand in the shortest possible time all those things which I have
learnt through danger and suffering in the course of many years. I have
not sought to adorn my work with long phrases or high-sounding words or
any of those allurements and ornaments with which many writers seek to
embellish their books, as I desire no honour for my work but such as
its truth and the gravity of its subject may justly deserve. Nor will
it, I trust, be deemed presumptuous on the part of a man of humble and
obscure condition to attempt to discuss and criticise the government of
princes; for in the same way that landscape painters station themselves
in the valleys in order to draw mountains or elevated ground, and
ascend an eminence in order to get a good view of the plains, so it
is necessary to be a prince to be able to know thoroughly the nature
of a people, and to know the nature of princes one must be one of the
populace.

May I trust, therefore, that Your Highness will accept this little gift
in the spirit in which it is offered; and if Your Highness will deign
to peruse it, you will recognise in it my ardent desire that you may
attain to that grandeur which fortune and your own merits presage for
you.

And should Your Highness gaze down from the summit of that eminence
towards this humble spot, you will recognise the great and unmerited
sufferings inflicted on me by a cruel fate.

\chapter{The Various Kinds of Government and the Ways by Which They Are Established}

All states and dominions which hold or have held sway over mankind are
either republics or monarchies. Monarchies are either hereditary ones,
in which the rulers have been for many years of the same family, or
else they are those of recent foundation. The newly founded ones are
either entirely new, as was Milan to Francesco Sforza, or else they
are, as it were, new members grafted on to the hereditary possessions
of the prince that annexes them, as is the kingdom of Naples to the
King of Spain. The dominions thus acquired have either been previously
accustomed to the rule of another prince, or else have been free
states, and they are annexed either by force of arms of the prince, or
of others, or else fall to him by good fortune or merit.

\chapter{Of Hereditary Monarchies}

I will not here speak of republics, having already treated of them
fully in another place. I will deal only with monarchies, and will show
how the various kinds described above can be governed and maintained.
In the first place, in hereditary states accustomed to the reigning
family the difficulty of maintaining them is far less than in new
monarchies; for it is sufficient not to exceed the ancestral usages,
and to accommodate one's self to accidental circumstances; in this way
such a prince, if of ordinary ability, will always be able to maintain
his position, unless some very exceptional and excessive force deprives
him of it; and even if he be thus deprived of it, on the slightest
misfortune happening to the new occupier, he will be able to regain it.

We have in Italy the example of the Duke of Ferrara, who was able
to withstand the assaults of the Venetians in the year '84, and of
Pope Julius in the year '10, for no other reason than because of the
antiquity of his family in that dominion. In as much as the legitimate
prince has less cause and less necessity to give offence, it is only
natural that he should be more loved; and, if no extraordinary vices
make him hated, it is only reasonable for his subjects to be naturally
attached to him, the memories and causes of innovations being forgotten
in the long period over which his rule has existed; whereas one change
always leaves the way prepared for the introduction of another.

\chapter{Of Mixed Monarchies}

But it is in the new monarchy that difficulties really exist. Firstly,
if it is not entirely new, but a member as it were of a mixed state,
its disorders spring at first from a natural difficulty which exists
in all new dominions, because men change masters willingly, hoping to
better themselves; and this belief makes them take arms against their
rulers, in which they are deceived, as experience shows them that they
have gone from bad to worse. This is the result of another very natural
cause, which is the necessary harm inflicted on those over whom the
prince obtains dominion, both by his soldiers and by an infinite number
of other injuries unavoidably caused by his occupation.

Thus you find enemies in all those whom you have injured by occupying
that dominion, and you cannot maintain the friendship of those who have
helped you to obtain this possession, as you will not be able to fulfil
their expectations, nor can you use strong measures with them, being
under an obligation to them; for which reason, however strong your
armies may be, you will always need the favour of the inhabitants to
take possession of a province. It was from these causes that Louis XII.
of France, though able to occupy Milan without trouble, immediately
lost it, and the forces of Ludovico alone were sufficient to take it
from him the first time, for the inhabitants who had willingly opened
their gates to him, finding themselves deluded in the hopes they had
cherished and not obtaining those benefits that they had anticipated,
could not bear the vexatious rule of their new prince.

It is indeed true that, after reconquering the rebel territories they
are not so easily lost again, for the ruler is now, by the fact of the
rebellion, less averse to secure his position by punishing offenders,
investigating any suspicious circumstances, and strengthening himself
in weak places. So that although the mere appearance of such a person
as Duke Ludovico on the frontier was sufficient to cause France to lose
Milan the first time, to make her lose her grip of it the second time
was only possible when all the world was against her, and after her
enemies had been defeated and driven out of Italy; which was the result
of the causes above mentioned. Nevertheless it was taken from her both
the first and the second time. The general causes of the first loss
have been already discussed; it remains now to be seen what were the
causes of the second loss and by what means France could have avoided
it, or what measures might have been taken by another ruler in that
position which were not taken by the King of France. Be it observed,
therefore, that those states which on annexation are united to a
previously existing state may or may not be of the same nationality
and language. If they are, it is very easy to hold them, especially if
they are not accustomed to freedom; and to possess them securely it
suffices that the family of the princes which formerly governed them
be extinct. For the rest, their old condition not being disturbed, and
there being no dissimilarity of customs, the people settle down quietly
under their new rulers, as is seen in the case of Burgundy, Brittany,
Gascony, and Normandy, which have been so long united to France; and
although there may be some slight differences of language, the customs
of the people are nevertheless similar, and they can get along well
together, and whoever obtains possession of them and wishes to retain
them must bear in mind two things: the one, that the blood of their old
rulers is extinct; the other, to make no alteration either in their
laws or in their taxes; in this way they will in a very short space of
time become united with their old possessions and form one state. But
when dominions are acquired in a province differing in language, laws,
and customs, the difficulties to be overcome are great, and it requires
good fortune as well as great industry to retain them; one of the best
and most certain means of doing so would be for the new ruler to take
up his residence in them. This would render their possession more
secure and durable, it is what the Turk has done in Greece; in spite of
all the other measures taken by him to hold that state, it would not
have been possible to retain it had he not gone to live there. Being
on the spot, disorders can be seen as they arise and can quickly be
remedied, but living at a distance, they are only heard of when they
get beyond remedy. Besides which, the province is not despoiled by
your officials, the subjects are pleased with the easy accessibility
of their prince; and wishing to be loyal they have more reason to love
him, and should they be otherwise they will have greater cause to fear
him.

Any external Power who wishes to assail that state will be less
disposed to do so; so that as long as he resides there he will be very
hard to dispossess. The other and better remedy is to plant colonies
in one or two of those places which form as it were the keys of the
land, for it is necessary either to do this or to maintain a large
force of armed men. The colonies will cost the prince little; with
little or no expense on his part, he can send and maintain them; he
only injures those whose lands and houses are taken to give to the new
inhabitants, and these form but a small proportion of the state, and
those who are injured, remaining poor and scattered, can never do any
harm to him, and all the others are, on the one hand, not injured and
therefore easily pacified; and, on the other, are fearful of offending
lest they should be treated like those who have been dispossessed of
their property. To conclude, these colonies cost nothing, are more
faithful, and give less offence; and the injured parties being poor
and scattered are unable to do mischief, as I have shown. For it must
be noted, that men must either be caressed or else annihilated; they
will revenge themselves for small injuries, but cannot do so for great
ones; the injury therefore that we do to a man must be such that we
need not fear his vengeance. But by maintaining a garrison instead of
colonists, one will spend much more, and consume in guarding it all
the revenues of that state, so that the acquisition will result in
a loss, besides giving much greater offence, since it injures every
one in that state with the quartering of the army on it; which being
an inconvenience felt, by all, every one becomes an enemy, and these
are enemies which can do mischief, as, though beaten, they remain in
their own homes. In every way, therefore, a garrison is as useless
as colonies are useful. Further, the ruler of a foreign province as
described, should make himself the leader and defender of his less
powerful neighbours, and endeavour to weaken the stronger ones, and
take care that his possessions are not entered by some foreigner not
less powerful than himself, who will always intervene at the request
of those who are discontented either through ambition or fear, as was
seen when the Ætoli invited the Romans into Greece; and in whatever
province they entered, it was always at the request of the inhabitants.
And the rule is that when a powerful foreigner enters a province, all
the less powerful inhabitants become his adherents, moved by the envy
they bear to those ruling over them; so much so that with regard to
these minor potentates he has no trouble whatever in winning them over,
for they willingly join forces with the state that he has acquired.
He has merely to be careful that they do not assume too much power
and authority, and he can easily with his own forces and their favour
put down those that are powerful and remain in everything the arbiter
of that province. And he who does not govern well in this way will
soon lose what he has acquired, and while he holds it will meet with
infinite difficulty and trouble.

The Romans in the provinces they took, always followed this policy;
they established colonies, flattered the less powerful without
increasing their strength, put down the most powerful and did not allow
foreign rulers to obtain influence in them. I will let the single
province of Greece suffice as an example. They made friends with the
Achæi and the Ætoli, the kingdom of Macedonia was cast down, and
Antiochus driven out, nor did they allow the merits of the Achæi or the
Ætoli to gain them any increase of territory, nor did the persuasions
of Philip induce them to befriend him without lowering him, nor could
the power of Antiochus make them consent to allow him to hold any state
in that province.

For the Romans did in this case what all wise princes should do, who
look not only at present dangers but also at future ones and diligently
guard against them; for being foreseen they can easily be remedied,
but if one waits till they are at hand, the medicine is no longer in
time as the malady has become incurable; it happening with this as with
those hectic fevers spoken of by doctors, which at their beginning
are easy to cure but difficult to recognise, but in course of time
when they have not at first been recognised and treated, become easy
to recognise and difficult to cure. Thus it happens in matters of
state; for knowing afar off (which it is only given to a prudent man
to do) the evils that are brewing, they are easily cured. But when,
for want of such knowledge, they are allowed to grow so that every
one can recognise them, there is no longer any remedy to be found.
However, the Romans, observing these disorders while yet remote, were
always able to find a remedy, and never allowed them to proceed in
order to avoid a war; for they knew that war was not to be avoided,
and could be deferred only to the advantage of the other side; they
therefore declared war against Philip and Antiochus in Greece, so as
not to have to fight them in Italy, though they might at the time
have avoided either; this they did not choose to do, never caring to
do that which is now every day to be heard in the mouths of our wise
men, to enjoy the benefits of time, but preferring those of their
own virtue and prudence, for time brings with it all things, and may
produce indifferently either good or evil. But let us return to France
and examine whether she did any of these things; and I will speak not
of Charles, but of Louis as the one whose proceedings can be better
seen, as he held possession in Italy for a longer time; you will then
see that he did the opposite of all those things which must be done to
keep possession of a foreign state. King Louis was called into Italy by
the ambition of the Venetians, who wished by his coming to gain half
of Lombardy. I will not blame the king for coming nor for the part
he took, because wishing to plant his foot in Italy, and not having
friends in the country, on the contrary the conduct of King Charles
having caused all doors to be closed to him, he was forced to accept
what friendships he could find, and his schemes would have quickly
been successful if he had made no mistakes in his other proceedings.

The king then having acquired Lombardy regained immediately the
reputation lost by Charles. Genoa yielded, the Florentines became his
friends, the Marquis of Mantua, the Dukes of Ferrara and Bentivogli,
the Lady of Furlì, the Lords of Faenza, Pesaro, Rimini, Camerino,
and Piombino, the inhabitants of Lucca, of Pisa, and of Sienna, all
approached him with offers of friendship. The Venetians might then
have seen the effects of their temerity, how to gain a few lands
in Lombardy they had made the king ruler over two-thirds of Italy.
Consider how little difficulty the king would have had in maintaining
his reputation in Italy if he had observed the rules above given, and
kept a firm and sure hold over all those friends of his, who being
many in number, and weak, and fearful one of the Church, another of
the Venetians, were always obliged to hold fast to him, and by whose
aid he could easily make sure of any who were still great. But he was
hardly in Milan before he did exactly the opposite, by giving aid to
Pope Alexander to occupy the Romagna. Nor did he perceive that, in
taking this course, he weakened himself, by casting off his friends
and those who had placed themselves at his disposal, and strengthened
the Church by adding to the spiritual power, which gives it such
authority, further temporal powers. And having made the first mistake,
he was obliged to follow it up, whilst, to put a stop to the ambition
of Alexander and prevent him becoming ruler of Tuscany, he was forced
to come to Italy. And not content with having increased the power
of the Church and lost his friends, he now desiring the kingdom of
Naples, divided it with the king of Spain; and where he alone was the
arbiter of Italy, he now brought in a companion, so that the ambitious
of that province who were dissatisfied with him might have some one
else to appeal to; and where he might have left in that kingdom a king
tributary to him, he dispossessed him in order to bring in another who
was capable of driving him out. The desire to acquire possessions is
a very natural and ordinary thing, and when those men do it who can
do so successfully, they are always praised and not blamed, but when
they cannot and yet want to do so at all costs, they make a mistake
deserving of great blame. If France, therefore, with her own forces
could have taken Naples, she ought to have done so; if she could not
she ought not to have divided it. And if the partition of Lombardy with
the Venetians is to be excused, as having been the means of allowing
the French king to set foot in Italy, this other partition deserves
blame, not having the excuse of necessity. Louis had thus made these
five mistakes: he had crushed the smaller Powers, increased the power
in Italy of one ruler, brought into the land a very powerful foreigner,
and he had not come to live there himself, nor had he established
any colonies. Still these mistakes might, if he had lived, not have
injured him, had he not made the sixth, that of taking the state from
the Venetians; for, if he had not strengthened the Church and brought
the Spaniards into Italy, it would have been right and necessary to
humble them; having once taken those measures, he ought never to have
consented to their ruin; because, had the Venetians been strong, it
would have kept the others from making attempts on Lombardy, partly
because the Venetians would not have consented to any measures by which
they did not get it for themselves, and partly because the others would
not have wanted to take it from France to give it to Venice, and would
not have had the courage to attack both.

If any one urges that King Louis yielded the Romagna to Alexander
and the kingdom to Spain in order to avoid war, I reply, with the
reasons already given, that one ought never to allow a disorder to take
place in order to avoid war, for war is not thereby avoided, but only
deferred to your disadvantage. And if others allege the promise given
by the king to the pope to undertake that enterprise for him, in return
for the dissolution of his marriage and for the cardinalship of Rohan,
I reply with what I shall say later on about the faith of princes and
how it is to be observed. Thus King Louis lost Lombardy through not
observing any of those conditions which have been observed by others
who have taken provinces and wished to retain them. Nor is this any
miracle, but very reasonable and natural. I spoke of this matter with
Cardinal Rohan at Nantes when Valentine, as Cesare Borgia, son of Pope
Alexander, was commonly called, was occupying the Romagna, for on
Cardinal Rohan saying to me that the Italians did not understand war,
I replied that the French did not understand politics, for if they did
they would never allow the Church to become so great. And experience
shows us that the greatness in Italy of the Church and also of Spain
have been caused by France, and her ruin has proceeded from them. From
which may be drawn a general rule, which never or very rarely fails,
that whoever is the cause of another becoming powerful, is ruined
himself; for that power is produced by him either through craft or
force; and both of these are suspected by the one that has become
powerful.

\chapter{Why the Kingdom of Darius, Occupied by Alexander, Did Not Rebel Against the Successors of the Latter After His Death}

Considering the difficulties there are in holding a newly acquired
state, some may wonder how it came to pass that Alexander the Great
became master of Asia in a few years, and had hardly occupied it
when he died, from which it might be supposed that the whole state
would have rebelled. However, his successors maintained themselves in
possession, and had no further difficulty in doing so than those which
arose among themselves from their own ambitions.

I reply that the kingdoms known to history have been governed in two
ways: either by a prince and his servants, who, as ministers by his
grace and permission, assist in governing the realm; or by a prince
and by barons, who hold their positions not by favour of the ruler but
by antiquity of blood. Such barons have states and subjects of their
own, who recognise them as their lords, and are naturally attached to
them. In those states which are governed by a prince and his servants,
the prince possesses more authority, because there is no one in the
state regarded as a superior besides himself, and if others are obeyed
it is merely as ministers and officials of the prince, and no one
regards them with any special affection. Examples of these two kinds
of government in our own time are the Turk and the King of France.
All the Turkish monarchy is governed by one ruler, the others are his
servants, and dividing his kingdom into "sangiacates," he sends to them
various administrators, and changes or recalls them at his pleasure.
But the King of France is surrounded by a large number of ancient
nobles, recognised as such by their subjects, and loved by them; they
have their prerogatives, which the king cannot deprive them of without
danger to himself. Whoever now considers these two states will see that
it would be difficult to acquire the state of the Turk; but having
conquered it, it would be very easy to hold it.

The causes of the difficulty of occupying the Turkish kingdom are, that
the invader could not be invited by princes of that kingdom, nor hope
to facilitate his enterprise by the rebellion of those around him, as
will be evident from reasons given above. Because, being all slaves,
and bound, it will be more difficult to corrupt them, and even if
they were corrupted, little effect could be hoped for, as they would
not be able to carry the people with them for the reasons mentioned.
Therefore, whoever assaults the Turk must be prepared to meet his
united forces, and must rely more on his own strength than on the
disorders of others; but having once conquered him, and beaten him in
battle so that he can no longer raise armies, nothing else is to be
feared except the family of the prince, and if this is extinguished,
there is no longer any one to be feared, the others having no credit
with the people; and as the victor before the victory could place no
hope in them, so he need not fear them afterwards. The contrary is
the case in kingdoms governed like that of France, because it is easy
to enter them by winning over some baron of the kingdom, there being
always some malcontents, and those desiring innovations. These can,
for the reasons stated, open the way to you and facilitate victory;
but afterwards, if you wish to keep possession, infinite difficulties
arise, both from those who have aided you and from those you have
oppressed. Nor is it sufficient to extinguish the family of the prince,
for there remain those nobles who will make themselves the head of new
changes, and being neither able to content them nor exterminate them,
you will lose the state whenever an occasion arises. Now if you will
consider what was the nature of the government of Darius you will find
it similar to the kingdom of the Turk, and therefore Alexander had
first to completely overthrow it and seize the country, after which
victory, Darius being dead, the state remained secure to Alexander,
for the reasons discussed above. And his successors, had they remained
united, might have enjoyed it in peace, nor did any tumults arise in
the kingdom except those fomented by themselves. But it is impossible
to possess with such ease countries constituted like France.

Hence arose the frequent rebellions of Spain, France, and Greece
against the Romans, owing to the numerous principalities which existed
in those states; for, as long as the memory of these lasted, the Romans
were always uncertain of their possessions; but when the memory of
these principalities had been extinguished they became, with the power
and duration of the empire, secure possessions.

And afterwards the latter could, when fighting among themselves, draw
each one with him a portion of these provinces, according to the
authority he had established there, and these provinces, when the
family of their ancient princes was extinct, recognised no other rulers
but the Romans. Considering these things, therefore, let no one be
surprised at the facility with which Alexander could hold Asia, and at
the difficulties that others have had in holding acquired possessions,
like Pyrrhus and many others; as this was not caused by the greater or
smaller ability of the conqueror, but depended on the dissimilarity of
the conditions.

\chapter{The Way to Govern Cities of Dominions That, Previous to Being Occupied, Lived Under Their Own Laws}

When those states which have been acquired are accustomed to live at
liberty under their own laws, there are three ways of holding them. The
first is to ruin them; the second is to go and live there in person;
the third is to allow them to live under their own laws, taking tribute
of them, and creating there within the country a state composed of a
few who will keep it friendly to you. Because this state, being created
by the prince, knows that it cannot exist without his friendship and
protection, and will do all it can to keep them, and a city used to
liberty can be more easily held by means of its citizens than in
any other way, if you wish to preserve it. There is the example of
the Spartans and the Romans. The Spartans held Athens and Thebes by
creating within them a state of a few people; nevertheless they lost
them. The Romans, in order to hold Capua, Carthage, and Numantia,
destroyed them, but did not lose them. They wanted to hold Greece in
almost the same way as the Spartans held it, leaving it free and under
its own laws, but they did not succeed; so that they were compelled
to destroy many cities in that province in order to keep it, because
in truth there is no sure method of holding them except by ruining
them. And whoever becomes the ruler of a free city and does not destroy
it, can expect to be destroyed by it, for it can always find a motive
for rebellion in the name of liberty and of its ancient usages, which
are forgotten neither by lapse of time nor by benefits received, and
whatever one does or provides, so long as the inhabitants are not
separated or dispersed, they do not forget that name and those usages,
but appeal to them at once in every emergency, as did Pisa after being
so many years held in servitude by the Florentines. But when cities or
provinces have been accustomed to live under a prince, and the family
of that prince is extinguished, being on the one hand used to obey, and
on the other not having their old prince, they cannot unite in choosing
one from among themselves, and they do not know how to live in freedom,
so that they are slower to take arms, and a prince can win them over
with greater facility and establish himself securely. But in republics
there is greater life, greater hatred, and more desire for vengeance;
they do not and cannot cast aside the memory of their ancient liberty,
so that the surest way is either to destroy them or reside in them.

\chapter{Of New Dominions Which Have Been Acquired by One's Own Arms and Powers}

Let no one marvel if in speaking of new dominions both as to prince
and state, I bring forward very exalted instances, for as men walk
almost always in the paths trodden by others, proceeding in their
actions by imitation, and not being always able to follow others
exactly, nor attain to the excellence of those they imitate, a prudent
man should always follow in the path trodden by great men and imitate
those who are most excellent, so that if he does not attain to their
greatness, at any rate he will get some tinge of it. He will do like
prudent archers, who when the place they wish to hit is too far off,
knowing how far their bow will carry, aim at a spot much higher than
the one they wish to hit, not in order to reach this height with
their arrow, but by help of this high aim to hit the spot they wish
to. I say then that in new dominions, where there is a new prince,
it is more or less easy to hold them according to the greater or
lesser ability of him who acquires them. And as the fact of a private
individual becoming a prince presupposes either great ability or good
fortune, it would appear that either of these things would mitigate
in part many difficulties. Nevertheless those who have been wanting
as regards good fortune have maintained themselves best. The matter
is also facilitated by the prince being obliged to reside personally
in his territory, having no others. But to come to those who have
become princes through their own merits and not by fortune, I regard
as the greatest, Moses, Cyrus, Romulus, Theseus, and such like. And
although one should not speak of Moses, he having merely carried out
what was ordered him by God, still he deserves admiration, if only
for that grace which made him worthy to speak with God. But regarding
Cyrus and others who have acquired or founded kingdoms, they will all
be found worthy of admiration; and if their particular actions and
methods are examined they will not appear very different from those of
Moses, although he had so great a Master. And in examining their life
and deeds it will be seen that they owed nothing to fortune but the
opportunity which gave them matter to be shaped into the form that they
thought fit; and without that opportunity their powers would have been
wasted, and without their powers the opportunity would have come in
vain. It was thus necessary that Moses should find the people of Israel
slaves in Egypt and oppressed by the Egyptians, so that they were
disposed to follow him in order to escape from their servitude. It was
necessary that Romulus should be unable to remain in Alba, and should
have been exposed at his birth, in order that he might become King of
Rome and founder of that nation. It was necessary that Cyrus should
find the Persians discontented with the empire of the Medes, and the
Medes weak and effeminate through long peace. Theseus could not have
showed his abilities if he had not found the Athenians dispersed.

These opportunities, therefore, gave these men their chance, and their
own great qualities enabled them to profit by them, so as to ennoble
their country and augment its fortunes. Those who by heroic means such
as these become princes, obtain their dominions with difficulty but
retain them easily, and the difficulties which they have in acquiring
their dominions arise in part from the new rules and regulations that
they have to introduce in order to establish their position securely.
It must be considered that there is nothing more difficult to carry
out, nor more doubtful of success, nor more dangerous to handle, than
to initiate a new order of things. For the reformer has enemies in all
those who profit by the old order, and only lukewarm defenders in all
those who would profit by the new order, this lukewarmness arising
partly from fear of their adversaries, who have the laws in their
favour; and partly from the incredulity of mankind, who do not truly
believe in anything new until they have had actual experience of it.
Thus it arises that on every opportunity for attacking the reformer,
his opponents do so with the zeal of partisans, the others only defend
him half-heartedly, so that between them he runs great danger. It is
necessary, however, in order to investigate thoroughly this question,
to examine whether these innovators are independent, or whether they
depend upon others, that is to say, whether in order to carry out
their designs they have to entreat or are able to force. In the first
case they invariably succeed ill, and accomplish nothing; but when
they can depend on their own strength and are able to use force, they
rarely fail. Thus it comes about that all armed prophets have conquered
and unarmed ones failed; for besides what has been already said, the
character of people varies, and it is easy to persuade them of a thing,
but difficult to keep them in that persuasion. And so it is necessary
to order things so that when they no longer believe, they can be made
to believe by force. Moses, Cyrus, Theseus, and Romulus would not have
been able to make their institutions observed for so long had they
been disarmed, as happened in our own time to Fra Girolamo Savonarola,
who failed entirely in his new rules when the multitude began to
disbelieve in him, and he had no means of holding fast those who had
believed nor of compelling the unbelievers to believe. Therefore
such men as these have great difficulty in making their way, and all
their dangers are met on the road and must be overcome by their own
abilities; but when once they have overcome them and have begun to be
held in veneration, and have suppressed those who envied them, they
remain powerful and secure, honoured and happy. To the high examples
given I will add a lesser one, which, however, is to be compared in
some measure with them and will serve as an instance of all such cases,
that of Jerone of Syracuse, who from a private individual became Prince
of Siracusa, without other aid from fortune beyond the opportunity;
for the Siracusans being oppressed elected him as their captain, from
which by merit he was made prince; while still in private life his
virtues were such that it was written of him, that he lacked nothing
to reign but the kingdom. He abolished the old militia, raised a new
one, abandoned his old friendships and formed new ones; and as he had
thus friends and soldiers of his own, he was able on this foundation
to build securely, so that while he had great trouble in acquiring his
position he had little in maintaining it.

\chapter{Of New Dominions Acquired By the Power of Others Or By Fortune}

Those who rise from private citizens to be princes merely by fortune
have little trouble in rising but very much in maintaining their
position. They meet with no difficulties on the way as they fly over
them, but all their difficulties arise when they are established. Such
are they who are granted a state either for money, or by favour of him
who grants it, as happened to many in Greece, in the cities of Ionia
and of the Hellespont, who were created princes by Darius in order to
hold these places for his security and glory; such were also those
emperors who from private citizens became emperors by bribing the
army. Such as these depend absolutely on the good will and fortune of
those who have raised them, both of which are extremely inconstant and
unstable. They neither know how to, nor are in a position to maintain
their rank, for unless he be a man of great genius it is not likely
that one who has always lived in a private position should know how to
command, and they are unable to command because they possess no forces
which will be friendly and faithful to them. Moreover, states quickly
founded, like all other things which are horn and grow rapidly, cannot
have deep roots, so that the first storm destroys them, unless, as
already said, the man who thus becomes a prince is of such great genius
as to be able to take immediate steps for maintaining what fortune
has thrown into his lap, and lay afterwards those foundations which
others make before becoming princes. With regard to these two methods
of becoming a prince,---by ability or by good fortune, I will here
adduce two examples which have taken place within our memory, those of
Francesco Sforza and Cesare Borgia.

Francesco, by appropriate means and through great abilities, from
citizen became Duke of Milan, and what he had attained after a thousand
difficulties he maintained with little trouble. On the other hand,
Cesare Borgia, commonly called Duke Valentine, acquired the state
through the fortune of his father and by the same means lost it, and
that although every measure was adopted by him and everything done
that a prudent and capable man could do to establish himself firmly
in that state that the arms and the favours of others had given him.
For, as we have said, he who does not lay his foundations beforehand
may by great abilities do so afterwards, although with great trouble
to the architect and danger to the building. If, then, one considers
the progress made by the duke, it will be seen how firm were the
foundations he had laid to his future power, which I do not think it
superfluous to examine, as I know of no better precepts for a new
prince to follow than the example of his actions; and if his measures
were not successful, it was through no fault of his own but only by
the most extraordinary malignity of fortune. In wishing to aggrandise
the duke his son, Alexander VI. had to meet very great difficulties
both present and future. In the first place, he saw no way of making
him ruler of any state that was not a possession of the Church. And in
attempting to take that of the Church, he knew that the Duke of Milan
and the Venetians would not consent, because Faenza and Rimini were
already under the protection of the Venetians. He saw, moreover, that
the arms of Italy, especially of those who might have served him, were
in the hands of those who would fear the greatness of the pope, and
therefore he could not depend upon them, being all under the Orsinis
and Colonnas and their adherents. It was, therefore, necessary to
disturb the existing condition and bring about disorders in the states
of Italy in order to obtain secure mastery over a part of them; this
was easy, for he found the Venetians, who, actuated by other motives,
had invited the French into Italy, which he not only did not oppose,
but facilitated by dissolving the marriage of King Louis. The king
came thus into Italy with the aid of the Venetians and the consent of
Alexander, and had hardly arrived at Milan before the pope obtained
troops from him for his enterprise in the Romagna, which he carried out
by means of the reputation of the king. The duke having thus obtained
the Romagna and defeated the Colonnas, was hindered in maintaining it
and proceeding further by two things: the one, his forces, of which he
doubted the fidelity; the other the will of France, that is to say,
he feared lest the arms of the Orsini of which he had availed himself
should fail him, and not only hinder him in obtaining more but take
from him what he had already conquered, and he also feared that the
king might do the same. He had evidence of this as regards the Orsini
when, after taking Faenza, he assaulted Bologna and observed their
backwardness in the assault. And as regards the king, he perceived his
designs when, after taking the dukedom of Urbino, he attacked Tuscany,
and the king made him desist from that enterprise; whereupon the
duke decided to depend no longer on the fortunes and arms of others.
The first thing he did was to weaken the parties of the Orsinis and
Colonnas in Rome by gaining all their adherents who were gentlemen and
making them followers of himself, by granting them large pensions,
and appointing them to commands and offices according to their rank,
so that their attachment to their parties was extinguished in a few
months, and entirely concentrated on the duke. After this he awaited an
opportunity for crushing the Orsinis, having dispersed the adherents of
the Colonna family, and when the opportunity arrived he made good use
of it, for the Orsini seeing at length that the greatness of the duke
and of the Church meant their own ruin, convoked a diet at Magione in
the Perugino. Hence sprang the rebellion of Urbino and the tumults in
Romagna and infinite dangers to the duke, who overcame them all with
the help of the French; and having regained his reputation, neither
trusting France nor other foreign forces in order not to have to oppose
them, he had recourse to stratagem. He dissembled his aims so well that
the Orsini, through the mediation of Signor Pavolo, made their peace
with him, which the duke spared no efforts to make secure, presenting
them with robes, money, and horses, so that in their simplicity they
were induced to come to Sinigaglia and fell into his hands. Having
thus suppressed these leaders and made their partisans his friends,
the duke had laid a very good foundation to his power, having all the
Romagna with the duchy of Urbino, and having gained the favour of
the inhabitants, who began to feel the benefit of his rule. And as
this part is worthy of note and of imitation by others, I will not
omit mention of it. When he took the Romagna, it had previously been
governed by weak rulers, who had rather despoiled their subjects than
governed them, and given them more cause for disunion than for union,
so that the province was a prey to robbery, assaults, and every kind
of disorder. He, therefore, judged it necessary to give them a good
government in order to make them peaceful and obedient to his rule.
For this purpose he appointed Messer Remiro d' Orco, a cruel and able
man, to whom he gave the fullest authority. This man, in a short time,
was highly successful in rendering the country orderly and united,
whereupon the duke, not deeming such excessive authority expedient,
lest it should become hateful, appointed a civil court of justice in
the middle of the province under an excellent president, to which each
city appointed its own advocate. And as he knew that the harshness of
the past had engendered some amount of hatred, in order to purge the
minds of the people and to win them over completely, he resolved to
show that if any cruelty had taken place it was not by his orders, but
through the harsh disposition of his minister. And taking him on some
pretext, he had him placed one morning in the public square at Cesena,
cut in half, with a piece of wood and blood-stained knife by his side.
The ferocity of this spectacle caused the people both satisfaction and
amazement. But to return to where we left off.

The duke being now powerful and partly secured against present perils,
being armed himself, and having in a great measure put down those
neighbouring forces which might injure him, had now to get the respect
of France, if he wished to proceed with his acquisitions, for he
knew that the king, who had lately discovered his error, would not
give him any help. He began therefore to seek fresh alliances and to
vacillate with France in the expedition that the French made towards
the kingdom of Naples against the Spaniards, who were besieging Gaeta.
His intention was to assure himself of them, which he would soon have
succeeded in doing if Alexander had lived. These were the measures
taken by him with regard to the present. As to the future, he feared
that a new successor to the Church might not be friendly to him and
might seek to deprive him of what Alexander had given him, and he
sought to provide against this in four ways. Firstly, by destroying all
who were of the blood of those ruling families which he had despoiled,
in order to deprive the pope of any opportunity. Secondly, by gaining
the friendship of the Roman nobles, so that he might through them hold
as it were the pope in check. Thirdly, by obtaining as great a hold on
the College as he could. Fourthly, by acquiring such power before the
pope died as to be able to resist alone the first onslaught. Of these
four things he had at the death of Alexander accomplished three, and
the fourth he had almost accomplished.

For of the dispossessed rulers he killed as many as he could lay hands
on, and very few escaped; he had gained to his party the Roman nobles;
and he had a great share in the College. As to new possessions, he
designed to become lord of Tuscany, and already possessed Perugia and
Piombino, and had assumed the protectorate over Pisa; and as he had
no longer to fear the French (for the French had been deprived of the
kingdom of Naples by the Spaniards in such a way that both parties
were obliged to buy his friendship) he seized Pisa. After this, Lucca
and Siena at once yielded, partly through envy of the Florentines and
partly through fear; the Florentines had no resources, so that, had he
succeeded as he had done before, in the very year that Alexander died
he would have gained such strength and renown as to be able to maintain
himself without depending on the fortunes or strength of others, but
solely by his own power and ability. But Alexander died five years
after he had first drawn his sword. He left him with the state of
Romagna only firmly established, and all the other schemes in mid-air,
between two very powerful and hostile armies, and suffering from a
fatal illness. But the valour and ability of the duke were such, and
he knew so well how to win over men or vanquish them, and so strong
were the foundations that he had laid in this short time, that if he
had not had those two armies upon him, or else had been in good health,
he would have survived every difficulty. And that his foundations were
good is seen from the fact that the Romagna waited for him more than a
month; in Rome, although half dead, he remained secure, and although
the Baglioni, Vitelli, and Orsini entered Rome they found no followers
against him. He was able, if not to make pope whom he wished, at any
rate to prevent a pope being created whom he did not wish. But if at
the death of Alexander he had been well everything would have been
easy. And he told me on the day that Pope Julius II. was created,
that he had thought of everything which might happen on the death of
his father, and provided against everything, except that he had never
thought that at his father's death he would be dying himself. Reviewing
thus all the actions of the duke, I find nothing to blame, on the
contrary, I feel bound, as I have done, to hold him up as an example
to be imitated by all who by fortune and with the arms of others have
risen to power. For with his great courage and high ambition he could
not have acted otherwise, and his designs were only frustrated by the
short life of Alexander and his own illness.

Whoever, therefore, deems it necessary in his new principality to
secure himself against enemies, to gain friends, to conquer by force
or fraud, to make himself beloved and feared by the people, followed
and reverenced by the soldiers, to destroy those who can and may injure
him, introduce innovations into old customs, to be severe and kind,
magnanimous and liberal, suppress the old militia, create a new one,
maintain the friendship of kings and princes in such a way that they
are glad to benefit him and fear to injure him, such a one can find no
better example than the actions of this man. The only thing he can be
accused of is that in the creation of Julius II. he made a bad choice;
for, as has been said, not being able to choose his own pope, he could
still prevent any one being made pope, and he ought never to have
permitted any of those cardinals to be raised to the papacy whom he had
injured, or who when pope would stand in fear of him. For men commit
injuries either through fear or through hate.

Those whom he had injured were, among others, San Pietro ad Vincula,
Colonna, San Giorgio, and Ascanio. All the others, if assumed to the
pontificate, would have had to fear him except Rohan and the Spaniards;
the latter through their relationship and obligations to him, the
former from his great power, being related to the King of France.
For these reasons the duke ought above all things to have created a
Spaniard pope; and if unable to, then he should have consented to Rohan
being appointed and not San Pietro ad Vincula. And whoever thinks that
in high personages new benefits cause old offences to be forgotten,
makes a great mistake. The duke, therefore, erred in this choice, and
it was the cause of his ultimate ruin.

\chapter{Of Those Who Have Attained the Position of Prince by Villainy}

But as there are still two ways of becoming prince which cannot be
attributed entirely either to fortune or to ability, they must not be
passed over, although one of them could be more fully discussed if we
were treating of republics. These are when one becomes prince by some
nefarious or villainous means, or when a private citizen becomes the
prince of his country through the favour of his fellow-citizens. And in
speaking of the former means, I will give two examples, one ancient,
the other modern, without entering further into the merits of this
method, as I judge them to be sufficient for any one obliged to imitate
them. Agathocles the Sicilian rose not only from private life but from
the lowest and most abject position to be King of Syracuse. The son
of a potter, he led a life of the utmost wickedness through all the
stages of his fortune. Nevertheless, his wickedness was accompanied
by such vigour of mind and body that, having joined the militia, he
rose through all its grades to be prætor of Syracuse. Having been
appointed to this position, and having decided to become prince, and
to hold with violence and without the support of others that which
had been granted him; and having imparted his design to Hamilcar the
Carthaginian, who with his armies was fighting in Sicily, he called
together one morning the people and senate of Syracuse, as if he had
to deliberate on matters of importance to the republic, and at a given
signal had all the senators and the richest men of the people killed by
his soldiers; after their death he occupied and held rule over the city
without any civil disorders. And although he was twice beaten by the
Carthaginians and ultimately besieged, he was able not only to defend
the city, but leaving a portion of his forces for its defence, with the
remainder he invaded Africa, and in a short time liberated Syracuse
from the siege and brought the Carthaginians to great extremities, so
that they were obliged to come to terms with him, and remain contented
with the possession of Africa, leaving Sicily to Agathocles. Whoever
considers, therefore, the actions and qualities of this man, will see
few if any things which can be attributed to fortune; for, as above
stated, it was not by the favour of any person, but through the grades
of the militia, which he had gained with a thousand hardships and
perils, that he arrived at the position of prince, which he afterwards
maintained by so many courageous and perilous expedients. It cannot be
called a virtue to kill one's fellow-citizens, betray one's friends,
be without faith, without pity, and without religion, by which methods
one may indeed gain an empire, but not glory. For if the virtues of
Agathocles in braving and overcoming perils, and his greatness of soul
in supporting and surmounting obstacles be considered, one sees no
reason for holding him inferior to any of the most renowned captains.
Nevertheless his barbarous cruelty and inhumanity, together with his
countless atrocities, do not permit of his being named among the
most famous men. We cannot attribute to fortune or merit that which
he achieved without either. In our own times, during the reign of
Alexander VI., Oliverotto du Fermo had been left a young boy under the
care of his maternal uncle, Giovanni Fogliani, who brought him up, and
sent him in early youth to fight under Paolo Vitelli, in order that he
might, under that discipline, obtain a good military position. On the
death of Paolo he fought under his brother Vitellozzo, and in a very
short time, being of great intelligence, and active in mind and body,
he became one of the leaders of his troops. But deeming it servile to
be under others, he resolved, with the help of some citizens of Fermo,
who preferred servitude to the liberty of their country, and with the
favour of the Vitellis, to occupy Fermo; he therefore wrote to Giovanni
Fogliani, how, having been for many years away from home, he wished
to come to see him and his city, and in some measure to revisit his
estates. And as he had only laboured to gain honour, in order that
his fellow-citizens might see that he had not spent his time in vain,
he wished to come honourably accompanied by one hundred horsemen, his
friends and followers, and prayed him that he would be pleased to
order that he should be received with honour by the citizens of Fermo,
by which he would honour not only him, Oliverotto, but also himself,
as he had been his pupil. Giovanni did not fail in any duty towards
his nephew; he caused him to be honourably received by the people of
Fermo, and lodged him in his own houses. After waiting some days to
arrange all that was necessary to his villainous projects, Oliverotto
invited Giovanni Fogliani and all the principal men of Fermo to a
grand banquet. After the dinner and the entertainments usual at such
feasts, Oliverotto artfully introduced certain important matters of
discussion, speaking of the greatness of Pope Alexander, and of his
son Cesare, and of their enterprises. To which discourses Giovanni
and others having replied, he all at once rose, saying that these
matters should be spoken of in a more secret place, and withdrew into
a room where Giovanni and the other citizens followed him. They were
no sooner seated than soldiers rushed out of hiding-places and killed
Giovanni and all the others. After which massacre Oliverotto mounted
his horse, rode through the town and besieged the chief magistrate in
his palace, so that through fear they were obliged to obey him and form
a government, of which he made himself prince. And all those being
dead who, if discontented, could injure him, he fortified himself
with new orders, civil and military, in such a way that within the
year that he held the principality he was not only safe himself in
the city of Fermo, but had become formidable to all his neighbours.
And his overthrow would have been difficult, like that of Agathocles,
if he had not allowed himself to be deceived by Cesare Borgia, when
he besieged the Orsinis and Vitellis at Sinigaglia, as already
related, where he also was taken, one year after the parricide he had
committed, and strangled, together with Vitellozzo, who had been his
teacher in ability and atrocity. Some may wonder how it came about
that Agathocles, and others like him, could, after infinite treachery
and cruelty, live secure for many years in their country and defend
themselves from external enemies without being conspired against by
their subjects; although many others have, through their cruelty, been
unable to maintain their position in times of peace, not to speak of
the uncertain times of war.

I believe this arises from the cruelties being used well or badly.
Well used may be called those (if it is permissible to use the word
well of evil) which are committed once for the need of securing one's
self, and which afterwards are not persisted in, but are exchanged for
measures as useful to the subjects as possible. Cruelties ill used
are those which, although at first few, increase rather than diminish
with time. Those who follow the former method may remedy in some
measure their condition, both with God and man; as did Agathocles.
As to the others, it is impossible for them to maintain themselves.
Whence it is to be noted, that in taking a state the conqueror must
arrange to commit all his cruelties at once, so as not to have to recur
to them every day, and so as to be able, by not making fresh changes,
to reassure people and win them over by benefiting them. Whoever acts
otherwise, either through timidity or bad counsels, is always obliged
to stand with knife in hand, and can never depend on his subjects,
because they, through continually fresh injuries, are unable to depend
upon him. For injuries should be done all together, so that being less
tasted, they will give less offence. Benefits should be granted little
by little, so that they may be better enjoyed. And above all, a prince
must live with his subjects in such a way that no accident should make
him change it, for good or evil; for necessity arising in adverse
times, you are not in time with severity, and the good that you do does
not profit you, as it is judged to be forced, and you will derive no
benefit whatever from it.

\chapter{Of the Civic Principality}

But we now come to the case where a citizen becomes prince not
through crime or intolerable violence, but by the favour of his
fellow-citizens, which may be called a civic principality. To arrive
at this position depends not entirely on worth or entirely on fortune,
but rather on cunning assisted by fortune. One attains it by help of
popular favour or by the favour of the aristocracy. For in every city
these two opposite parties are to be found, arising from the desire
of the populace to avoid the oppression of the great, and the desire
of the great to command and oppress the people. And from these two
opposing interests arises in the city one of three effects: either
absolute government, liberty, or license. The former is created either
by the populace or the nobility depending on the relative opportunities
of the two parties; for when the nobility see that they are unable to
resist the people they unite in creating one of their number prince,
so as to be able to carry out their own designs under the shadow of
his authority. The populace, on the other hand, when unable to resist
the nobility, endeavour to create a prince in order to be protected
by his authority. He who becomes prince by help of the nobility has
greater difficulty in maintaining his power than he who is raised by
the populace, for he is surrounded by those who think themselves his
equals, and is thus unable to direct or command as he pleases. But one
who is raised to the leadership by popular favour finds himself alone,
and has no one or very few who are not ready to obey him. Besides
which, it is impossible to satisfy the nobility by fair dealing and
without inflicting injury on others, whereas it is very easy to satisfy
the mass of the people in this way. For the aim of the people is more
honest than that of the nobility, the latter desiring to oppress, and
the former merely to avoid oppression. It must also be added that the
prince can never insure himself against a hostile populace on account
of their number, but he can against the hostility of the great, as
they are but few. The worst that a prince has to expect from a hostile
people is to be abandoned, but from hostile nobles he has to fear not
only abandonment but their active opposition, and as they are more
farseeing and more cunning, they are always in time to save themselves
and take sides with the one who they expect will conquer. The prince
is, moreover, obliged to live always with the same people, but he can
easily do without the same nobility, being able to make and unmake them
at any time, and increase their position or deprive them of it as he
pleases. And to throw further light on this part, I would say, that the
nobles are to be considered in two different manners; that is, they are
either to be ruled so as to make them entirely depend on your fortunes,
or else not. Those that are thus bound to you and are not rapacious,
must be honoured and loved; those who are not bound must be considered
in two ways, they either do this through pusillanimity and natural
want of courage, and in this case you ought to make use of them, and
especially such as are of good counsel, so that they may honour you in
prosperity and in adversity you have not to fear them. But when they
are not bound to you of set purpose and for ambitious ends, it is a
sign that they think more of themselves than of you; and from such men
the prince must guard himself and look upon them as secret enemies,
who will help to ruin him when in adversity. One, however, who becomes
prince by favour of the populace, must maintain its friendship, which
he will find easy, the people asking nothing but not to be oppressed.
But one who against the people's wishes becomes prince by favour of
the nobles, should above all endeavour to gain the favour of the
people; this will be easy to him if he protects them. And as men, who
receive good from those they expected evil from, feel under a greater
obligation to their benefactor, so the subject populace will become
even better disposed towards him than if he had become prince through
their favour. The prince can win their favour in many ways, which vary
according to circumstances, for which no certain rule can be given, and
will therefore be passed over.

I will only say, in conclusion, that it is necessary for a prince to
possess the friendship of the people; otherwise he has no resource in
times of adversity. Nabis, prince of the Spartans, sustained a siege by
the whole of Greece and a victorious Roman army, and defended against
them his country and maintained his own position. It sufficed when the
danger arose for him to make sure of a few, which would not have been
enough if the populace had been hostile to him. And let no one oppose
my opinion in this by quoting the trite proverb, "He who builds on the
people, builds on mud"; because that is true when a private citizen
relies upon the people and persuades himself that they will liberate
him if he is oppressed by enemies or by the magistrates; in this case
he might often find himself deceived, as happened in Rome to the
Gracchi and in Florence to Messer Georgio Scali.

But when it is a prince who founds himself on this basis, one who
can command and is a man of courage, and does not get frightened in
adversity, and does not neglect other preparations, and one who by his
own courage and measures animates the mass of the people, he will not
find himself deceived by them, and he will find that he has laid his
foundations well. Usually these principalities are in danger when the
prince from the position of a civil ruler changes to an absolute one,
for these princes either command themselves or by means of magistrates.
In the latter case their position is weaker and more dangerous, for
they are at the mercy of those citizens who are appointed magistrates,
who can, especially in times of adversity, with great facility deprive
them of their position, either by acting against them or by not obeying
them. The prince is not in time, in such dangers, to assume absolute
authority, for the citizens and subjects who are accustomed to take
their orders from the magistrates are not ready in these emergencies
to obey his, and he will always in doubtful times lack men whom he can
rely on. Such a prince cannot base himself on what he sees in quiet
times, when the citizens have need of the state; for then every one is
full of promises and each one is ready to die for him when death is
far off; but in adversity, when the state has need of citizens, then
he will find but few. And this experience is the more dangerous, in
that it can only be had once. Therefore a wise prince will seek means
by which his subjects will always and in every possible condition
of things have need of his government, and then they will always be
faithful to him.

\chapter{How the Strength of All States Should be Measured}

In examining the character of these principalities it is necessary to
consider another point, namely, whether the prince has such a position
as to be able in case of need to maintain himself alone, or whether he
has always need of the protection of others. The better to explain this
I would say, that I consider those capable of maintaining themselves
alone who can, through abundance of men or money, put together a
sufficient army, and hold the field against any one who assails them;
and I consider to have need of others, those who cannot take the
field against their enemies, but are obliged to take refuge within
their walls and stand on the defensive. We have already discussed the
former case and will speak in future of it as occasion arises. In
the second case there is nothing to be said except to encourage such
a prince to provision and fortify his own town, and not to trouble
about the country. And whoever has strongly fortified his town and,
as regards the government of his subjects, has proceeded as we have
already described and will further relate, will be attacked with great
reluctance, for men are always averse to enterprises in which they
foresee difficulties, and it can never appear easy to attack one who
has his town well guarded and is not hated by the people. The cities
of Germany are extremely liberal, have little surrounding country, and
obey the emperor when they choose, and they do not fear him or any
other potentate that they have about them. They are fortified in such
a manner that every one thinks that to reduce them would be tedious
and difficult, for they all have the necessary moats and bastions,
sufficient artillery, and always keep in the public storehouses food
and drink and fuel for one year. Beyond which, to keep the lower
classes satisfied, and without loss to the public, they have always
enough means to give them work for one year in these employments which
form the nerve and life of the town, and in the industries by which the
lower classes live; military exercises are still held in reputation,
and many regulations are in force for maintaining them. A prince,
therefore, who possesses a strong city and does not make himself hated,
cannot be assaulted; and if he were to be so, the assailant would
be obliged to retire shamefully; for so many things change, that it
is almost impossible for any one to hold the field for a year with
his armies idle. And to those who urge that the people, having their
possessions outside and seeing them burnt, will not have patience, and
the long siege and self-interest will make them forget their prince,
I reply that a powerful and courageous prince will always overcome
those difficulties by now raising the hopes of his subjects that the
evils will not last long, now impressing them with fear of the enemy's
cruelty, now by dextrously assuring himself of those who appear too
bold. Besides which, the enemy would naturally burn and ruin the
country on first arriving and in the time when men's minds are still
hot and eager to defend themselves, and therefore the prince has still
less to fear, for after some days, when people have cooled down, the
damage is done, the evil has been suffered, and there is no remedy, so
that they are the more ready to unite with their prince, as it appears
that he is under an obligation to them, their houses having been burnt
and their possessions ruined in his defence.

It is the nature of men to be us much bound by the benefits that they
confer as by those they receive. From which it follows that, everything
considered, a prudent prince will not find it difficult to uphold the
courage of his subjects both at the commencement and during a state of
siege, if he possesses provisions and means to defend himself.

\chapter{Of Ecclesiastical Principalities}

It now remains to us only to speak of ecclesiastical principalities,
with regard to which the difficulties lie wholly before they are
possessed. They are acquired either by ability or by fortune; but
are maintained without either, for they are sustained by the ancient
religious customs, which are so powerful and of such quality, that they
keep their princes in power in whatever manner they proceed and live.
These alone have a state without defending it, have subjects without
governing them, and the states, not being defended, are not taken from
them; the subjects not being governed do not disturb themselves, and
neither think of nor are capable of alienating themselves from them.
Only these principalities, therefore, are secure and happy. But as they
are upheld by higher causes, which the human mind cannot attain to, I
will abstain from speaking of them; for being exalted and maintained by
God, it would be the work of a presumptuous and foolish man to discuss
them.

However, I might be asked how it has come about that the Church has
reached such great temporal power, when, previous to Alexander VI.,
the Italian potentates,---and not merely the really powerful ones, but
every lord or baron, however insignificant, held it in slight esteem
as regards temporal power; whereas now it is dreaded by a king of
France, whom it has been able to drive out of Italy, and has also been
able to ruin the Venetians. Therefore, although this is well known, I
do not think it superfluous to call it to mind. Before Charles, King
of France, came into Italy, this country was under the rule of the
pope, the Venetians, the King of Naples, the Duke of Milan, and the
Florentines. These potentates had to have two chief cares: one, that no
foreigner should enter Italy by force of arms, the other that none of
the existing governments should extend its dominions. Those chiefly to
be watched were the pope and the Venetians. To keep back the Venetians
required the ruin of all the others, as in the defence of Ferrara,
and to keep down the pope they made use of the Roman barons. These
were divided into two factions, the Orsinis and the Colonnas, and as
there was constant quarrelling between them, and they were constantly
under arms, before the eyes of the pope, they kept the papacy weak and
infirm. And although there arose now and then a resolute pope like
Sextus, yet his fortune or ability was never able to liberate him from
these evils. The shortness of their life was the reason of this, for
in the course of ten years which, as a general rule, a pope lived, he
had great difficulty in suppressing even one of the factions, and if,
for example, a pope had almost put down the Colonnas, a new pope would
succeed who was hostile to the Orsinis, which caused the Colonnas to
spring up again, and he was not in time to suppress them. This caused
the temporal power of the pope to be of little esteem in Italy.

Then arose Alexander VI. who of all the pontiffs who have ever
reigned, best showed how a pope might prevail both by money and by
force. With Duke Valentine as his instrument, and on the occasion of
the French invasion, he did all that I have previously described
in speaking of the actions of the duke. And although his object was
to aggrandise not the Church but the duke, what he did resulted in
the aggrandisement of the Church, which after the death of the duke
became the heir of his labours. Then came Pope Julius, who found
the Church powerful, possessing all Romagna, all the Roman barons
suppressed, and the factions destroyed by the severity of Alexander.
He also found the way open for accumulating wealth in ways never used
before the time of Alexander. These measures were not only followed
by Julius, but increased; he resolved to gain Bologna, put down the
Venetians and drive the French from Italy, in all which enterprises
he was successful. He merits the greater praise, as he did everything
to increase the power of the Church and not of any private person. He
also kept the Orsini and Colonna parties in the conditions in which he
found them, and although there were some leaders among them who might
have made changes, there were two things that kept them steady: one,
the greatness of the Church, which they dreaded; the other, the fact
that they had no cardinals, who are the origin of the tumults among
them. For these parties are never at rest when they have cardinals,
for these stir up the parties both within Rome and outside, and the
barons are forced to defend them. Thus from the ambitions of prelates
arise the discords and tumults among the barons. His holiness, Pope Leo
X., therefore, has found the pontificate in a very powerful condition,
from which it is hoped, that as those popes made it great by force of
armies, so he through his goodness and infinite other virtues will make
it both great and venerated.

\chapter{The Different Kinds of Militia and Mercenary Soldiers}

Having now discussed fully the qualities of these principalities of
which I proposed to treat, and partially considered the causes of their
prosperity or failure, and having also showed the methods by which
many have sought to obtain such states, it now remains for me to treat
generally of the methods of attack and defence that can be used in
each of them. We have said already how necessary it is for a prince
to have his foundations good, otherwise he is certain to be ruined.
The chief foundations of all states, whether new, old, or mixed, are
good laws and good arms. And as there cannot be good laws where there
are not good arms, and where there are good arms there should be good
laws, I will not now discuss the laws, but will speak of the arms. I
say, therefore, that the arms by which a prince defends his possessions
are either his own, or else mercenaries, or auxiliaries, or mixed. The
mercenaries and auxiliaries are useless and dangerous, and if any one
keeps his state based on the arms of mercenaries, he will never stand
firm or sure, as they are disunited, ambitious, without discipline,
faithless, bold amongst friends, cowardly amongst enemies, they have no
fear of God, and keep no faith with men. Ruin is only deferred as long
as the assault is postponed; in peace you are despoiled by them, and in
war by the enemy. The cause of this is that they have no love or other
motive to keep them in the field beyond a trifling wage, which is not
enough to make them ready to die for you. They are quite willing to be
your soldiers so long as you do not make war, but when war comes, it is
either fly or be off. I ought to have little trouble in proving this,
since the ruin of Italy is now caused by nothing else but through her
having relied for many years on mercenary arms. These were somewhat
improved in a few cases, and appeared courageous among themselves, but
when the foreigner came they showed their worthlessness. Thus it came
about that King Charles of France was allowed to take Italy without the
slightest trouble, and those who said that it was owing to our sins,
spoke the truth, but it was not the sins that they believed but those
that I have related. And as it was the sins of princes, they too have
suffered the punishment. I will explain more fully the defects of these
arms. Mercenary captains are either very capable men or not; if they
are, you cannot rely upon them, for they will always aspire to their
own greatness, either by oppressing you, their master, or by oppressing
others against your intentions; but if the captain is not an able man,
he will generally ruin you. And if it is replied to this, that whoever
has armed forces will do the same, whether these are mercenary or not,
I would reply that as armies are to be used either by a prince or by a
republic, the prince must go in person to take the position of captain,
and the republic must send its own citizens. If the one sent turns,
out incompetent, it must change him; and if capable, keep him by law
from going beyond the proper limits. And it is seen by experience that
only princes and armed republics make very great progress, whereas
mercenary forces do nothing but damage, and also an armed republic
submits less easily to the rule of one of its citizens than a republic
armed by foreign forces. Rome and Sparta were for many centuries well
armed and free. The Swiss are well armed and enjoy great freedom. As an
example of mercenary armies in antiquity there are the Carthaginians,
who were oppressed by their mercenary soldiers, after the termination
of the first war with the Romans, even while they still had their own
citizens as captains. Philip of Macedon was made captain of their
forces by the Thebans after the death of Epaminondas, and after gaining
the victory he deprived them of liberty. The Milanese, on the death
of Duke Philip, hired Francesco Sforza against the Venetians, who
having overcome the enemy at Caravaggio, allied himself with them to
oppress the Milanese his employers. The father of this Sforza, being
a soldier in the service of the Queen Giovanna of Naples, left her
suddenly unarmed, by which she was compelled, in order not to lose the
kingdom, to throw herself into the arms of the King of Aragon. And
if the Venetians and Florentines have in times past increased their
dominions by means of such forces, and their captains have not made
themselves princes but have defended them, I reply that the Florentines
in this case have been favoured by chance, for of the capable leaders
whom they might have feared, some did not conquer, some met with
opposition, and others directed their ambition elsewhere. The one who
did not conquer was Sir John Hawkwood, whose fidelity could not be
known as he was not victorious, but every one will admit that, had he
conquered, the Florentines would have been at his mercy. Sforza had
always the Bracceschi against him, they being constantly at enmity.
Francesco directed his ambition towards Lombardy; Braccio against the
Church and the kingdom of Naples. But let us look at what followed a
short time ago. The Florentines appointed Paolo Vitelli their captain,
a man of great prudence, who had risen from a private station to the
highest reputation. If he had taken Pisa no one can deny that it was
highly important for the Florentines to retain his friendship, because
had he become the soldier of their enemies they would have had no
means of opposing him; and in order to retain him they would have been
obliged to obey him. As to the Venetians, if one considers the progress
they made, it will be seen that they acted surely and gloriously so
long as they made war with their own forces; that it was before they
commenced their enterprises on land that they fought courageously with
their own gentlemen and armed populace, but when they began to fight
on land they abandoned this virtue, and began to follow the Italian
custom. And at the commencement of their land conquests they had not
much to fear from their captains, their land possessions not being
very large, and their reputation being great, but as their possessions
increased, as they did under Carmagnola, they had an example of their
mistake. For seeing that he was very powerful, after he had defeated
the Duke of Milan, and knowing, on the other hand, that he was not
enterprising in warfare, they considered that they would not make any
more conquests with him, and they neither would nor could dismiss him,
for fear of losing what they had already gained. They were therefore
obliged, in order to make sure of him, to have him killed. They then
had for captains Bartolommeo da Bergamo, Roberto da San Severino, Count
di Pitigliano, and such like, from whom they had to fear loss instead
of gain, as happened subsequently at Vailà, where in one day they lost
what they had laboriously gained in eight hundred years; for with these
forces, only slow and trifling acquisitions are made, but sudden and
miraculous losses. And as I have cited these examples from Italy,
which has now for many years been governed by mercenary forces, I will
now deal more largely with them, so that having seen their origin and
progress, they can be better remedied. You must understand that in
these latter times, as soon as the empire began to be repudiated in
Italy and the pope to gain greater reputation in temporal matters,
Italy was divided into many states; many of the principal cities took
up arms against their nobles, who, favoured by the emperor, had held
them in subjection, and the Church encouraged this in order to increase
its temporal power. In many other cities one of the inhabitants became
prince. Thus Italy having fallen almost entirely into the hands of
the Church and a few republics, and the priests and other citizens
not being accustomed to bear arms, they began to hire foreigners as
soldiers. The first to bring reputation for this kind of militia was
Alberigo da Como, a native of Romagna. The discipline of this man
produced, among others, Braccio and Sforza, who were in their day the
arbiters of Italy. After these came all those others who up to the
present day have commanded the armies of Italy, and the result of their
prowess has been that Italy has been overrun by Charles, preyed on by
Louis, tyrannised over by Ferrando, and insulted by the Swiss. The
system adopted by them was, in the first place, to increase their own
reputation by discrediting the infantry. They did this because, as they
had no country and lived on their earnings, a few foot soldiers did not
augment their reputation, and they could not maintain a large number
and therefore they restricted themselves almost entirely to cavalry,
by which with a smaller number they were well paid and honoured. They
reduced things to such a state that in an army of 20,000 soldiers there
were not 2000 foot. They had also used every means to spare themselves
and the soldiers any hardship or fear by not killing each other in
their encounters, but taking prisoners without a blow. They made no
attacks on fortifications by night; and those in the fortifications did
not attack the tents at night, they made no stockades on ditches round
their camps, and did not take the field in winter. All these things
were permitted by their military rules, and adopted, as we have said,
to avoid trouble and danger, so that they have reduced Italy to slavery
and degradation.

\chapter{Of Auxiliary, Mixed, and Native Troops}

Auxiliary forces, which are the other kind of useless forces, are
when one calls on a potentate to come and aid one with his troops, as
was done in recent times by Julius, who seeing the wretched failure
of his mercenary forces, in his Ferrara enterprise, had recourse to
auxiliaries, and arranged with Ferrando, King of Spain, that he should
help him with his armies. These forces may be good in themselves, but
they are always dangerous for those who borrow them, for if they lose
you are defeated, and if they conquer you remain their prisoner. And
although ancient history is full of examples of this, I will not depart
from the example of Pope Julius II., which is still fresh. Nothing
could be less prudent than the course he adopted; for, wishing to take
Ferrara, he put himself entirely into the power of a foreigner. But
by good fortune there arose a third cause which prevented him reaping
the effects of his bad choice; for when his auxiliaries were beaten
at Ravenna, the Swiss rose up and drove back the victors, against all
expectation of himself or others, so that he was not taken prisoner by
the enemy which had fled, nor by his own auxiliaries, having conquered
by other arms than theirs. The Florentines, being totally disarmed,
hired 10,000 Frenchmen to attack Pisa, by which measure they ran
greater risk than at any period of their struggles. The emperor of
Constantinople, to oppose his neighbours, put 10,000 Turks into Greece,
who after the war would not go away again, which was the beginning of
the servitude of Greece to the infidels. Any one, therefore, who wishes
not to conquer, would do well to use these forces, which are much more
dangerous than mercenaries, as with them ruin is complete, for they
are all united, and owe obedience to others, whereas with mercenaries,
when they have conquered, it requires more time and a good opportunity
for them to injure you, as they do not form a single body and have been
engaged and paid by you, therefore a third party that you have made
leader cannot at once acquire enough authority to be able to injure
you. In a word, the greatest dangers with mercenaries lies in their
cowardice and reluctance to fight, but with auxiliaries the danger
lies in their courage. A wise prince, therefore, always avoids these
forces and has recourse to his own, and would prefer rather to lose
with his own men than conquer with the forces of others, not deeming
it a true victory which is gained by foreign arms. I never hesitate to
cite the example of Cesare Borgia and his actions. This duke entered
Romagna with auxiliary troops, leading forces composed entirely of
French soldiers, and with these he took Imola and Forli; but as they
seemed unsafe, he had recourse to mercenaries, and hired the Orsini and
Vitelli; afterwards finding these uncertain to handle, unfaithful and
dangerous, he suppressed them, and relied upon his own men. And the
difference between these forces can be easily seen if one considers
the difference between the reputation of the duke when he had only the
French, when he had the Orsini and Vitelli, and when he had to rely
on himself and his own soldiers. His reputation will be found to have
constantly increased, and he was never so highly esteemed as when
every one saw that he was the sole master of his forces.

I do not wish to go away from recent Italian instances, but I cannot
omit Hiero of Syracuse, whom I have already mentioned. This man being,
as I said, made head of the army by the Syracusans, immediately
recognised the uselessness of that mercenary militia which was composed
like our Italian mercenary troops, and as he thought it unsafe
either to retain them or dismiss them, he had them cut in pieces and
thenceforward made war with his own arms and not those of others. I
would also call to mind a figure out of the Old Testament which well
illustrates this point. When David offered to Saul to go and fight with
the Philistine champion Goliath, Saul, to encourage him, armed him
with his own arms, which when David had tried on he refused saying,
that with them he could not fight so well; he preferred, therefore,
to face the enemy with his own sling and knife. In short, the arms of
others either fall away from you, or overburden you, or else impede
you. Charles VIII., father of King Louis XL, having through good
fortune and bravery liberated France from the English, recognised this
necessity of being armed with his own forces, and established in his
kingdom a system of men-at-arms and infantry. Afterwards King Louis
his son abolished the infantry and began to hire Swiss, which mistake
being followed by others is, as may now be seen, a cause of danger to
that kingdom. For by giving such reputation to the Swiss, France has
disheartened all her own troops, the infantry having been abolished and
the men-at-arms being obliged to foreigners for assistance; for being
accustomed to fight with Swiss troops, they think they cannot conquer
without them. Whence it comes that the French are insufficiently strong
to oppose the Swiss, and without the aid of the Swiss they will not
venture against others. The armies of the French are thus of a mixed
kind, partly mercenary and partly her own; taken together they are much
better than troops entirely composed of mercenaries or auxiliaries, but
much inferior to national forces.

\chapter{What the Duties of a Prince Are with Regard to the Militia}

A Prince should therefore have no other aim or thought, nor take up
any other thing for his study, but war and its order and discipline,
for that is the only art that is necessary to one who commands, and
it is of such virtue that it not only maintains those who are born
princes, but often enables men of private fortune to attain to that
rank. And one sees, on the other hand, that when princes think more of
luxury than of arms, they lose their state. The chief cause which makes
any one lose it, is the contempt of this art, and the way to acquire
it is to be well versed in the same. Francesco Sforza, through being
well armed, became, from a private position, Duke of Milan; his sons,
through wishing to avoid the fatigue and hardship of war, from dukes
became private persons. For among other evils caused by being disarmed,
it renders you contemptible; which is one of those disgraceful things
which a prince must guard against, as will be explained later. Because
there is no comparison whatever between an armed man and a disarmed
one; it is not reasonable to suppose that one who is armed will obey
willingly one who is unarmed; or that any unarmed man will remain
safe among armed servants. For one being disdainful and the other
suspicious, it is not possible for them to act well together. And
yet a prince who is ignorant of military matters, besides the other
misfortunes already mentioned, cannot be esteemed by his soldiers, nor
have confidence in them. He ought, therefore, never to let his thoughts
stray from the exercise of war; and in peace he ought to practise it
more than in war, which he can do in two ways: both by action and by
study. As to action, he must, besides keeping his men well disciplined
and exercised, engage continually in hunting, and thus accustom his
body to hardships; and on the other hand learn the nature of the
land, how the mountains rise, how the valleys are disposed, where the
plains lie, and understand the nature of the rivers and swamps, and
to this he should devote great attention. This knowledge is useful in
two ways. In the first place, one learns to know one's country, and
can the better see how to defend it. Then by means of the knowledge
and experience gained in one locality, one can easily understand
any other that it may be necessary to venture on, for the hills and
valleys, plains and rivers of Tuscany, for instance, have a certain
resemblance to those of other provinces, so that from a knowledge of
the country in one province one can easily arrive at a knowledge of
others. And that prince who is lacking in this skill is wanting in the
first essentials of a leader; for it is this which teaches how to find
the enemy, take up quarters, lead armies, arrange marches and occupy
positions with advantage. Philopœmen, prince of the Achæi, among other
praises bestowed on him by writers, is lauded because in times of peace
he thought of nothing but the methods of warfare, and when he was in
the country with his friends, he often stopped and asked them: If the
enemy were on that hill and we found ourselves here with our army,
which of us would have the advantage? How could we safely approach
him maintaining our order? If we wished to retire, what ought we to
do? If they retired, how should we follow them? And he put before them
as they went along all the cases that might happen to an army, heard
their opinion, gave his own, fortifying it by argument; so that through
these continued cogitations there could never happen any incident when
leading his armies for which he was not prepared. But as to exercise
for the mind, the prince ought to read history and study the actions
of eminent men, see how they acted in warfare, examine the causes of
their victories and losses in order to imitate the former and avoid the
latter, and above all, do as some eminent men have done in the past,
who have imitated some one, who has been much praised and glorified,
and have always kept their deeds and actions before them, as they say
Alexander the Great imitated Achilles, Cæsar Alexander, and Scipio
Cyrus. And whoever reads the life of Cyrus written by Xenophon, will
perceive in the life of Scipio how gloriously he imitated him, and how,
in chastity, affability, humanity, and liberality Scipio conformed to
those qualities of Cyrus described by Xenophon.

A wise prince should follow similar methods and never remain idle in
peaceful times, but by industry make such good use of the time as may
serve him in adversity, so that when fortune changes she may find him
prepared to resist her blows.

\chapter{Of the Things for Which Men, and Especially Princes, Are Praised or Blamed}

It remains now to be seen what are the methods and rules for a
prince as regards his subjects and friends. And as I know that many
have written of this, I fear that my writing about it may be deemed
presumptuous, differing as I do, especially in this matter, from the
opinions of others. But my intention being to write something of use to
those who understand it, it appears to me more proper to go to the real
truth of the matter than to its imagination; and many have imagined
republics and principalities which have never been seen or known to
exist in reality; for how we live is so far removed from how we ought
to live, that he who abandons what is done for what ought to be done,
will rather learn to bring about his own ruin than his preservation.
A man who wishes to make a profession of goodness in everything must
necessarily come to grief among so many who are not good. Therefore it
is necessary for a prince, who wishes to maintain himself, to learn how
not to be good, and to use it and not use it according to the necessity
of the case. Leaving on one side then those things which concern only
an imaginary prince, and speaking of those that are real, I state that
all men, when spoken of, and especially princes, who are placed at
a greater height, are noted for some of those qualities which bring
them either praise or blame. Thus one is considered liberal, another
miserly; one a free giver, another rapacious; one cruel, another
merciful; one a breaker of his word, another faithful; one effeminate
and pusillanimous, another fierce and high-spirited; one humane,
another proud; one lascivious, another chaste; one frank, another
astute; one hard, another easy; one serious, another frivolous; one
religious, another incredulous, and so on. I know that every one will
admit that it would be highly praiseworthy in a prince to possess all
the above-named qualities that are reputed good, but as they cannot all
be possessed or observed, human conditions not permitting of it, it is
necessary that he should be prudent enough to avoid the disgrace of
those vices which would lose him the state, and guard himself against
those which will not lose it him, if possible, but if not able to, he
can indulge them with less scruple. And yet he must not mind incurring
the disgrace of those vices, without which it would be difficult to
save the state, for if one considers well, it will be found that some
things which seem virtues would, if followed, lead to one's ruin, and
some others which appear vices result, if followed, in one's greater
security and well being.

\chapter{Of Liberality and Niggardliness}

Beginning now with the first qualities above named, I say that it
would be well to be considered liberal; nevertheless liberality used
in such a way that you are not feared will injure you, because if
used virtuously and in the proper way, it will not be known, and you
will not incur the disgrace of the contrary vice. But one who wishes
to obtain the reputation of liberality among men, must not omit every
kind of sumptuous display, and to such an extent that a prince of this
character will consume by such means all his resources, and will be
at last compelled, if he wishes to maintain his name for liberality,
to impose heavy charges on his people, become an extortioner, and do
everything possible to obtain money. This will make his subjects begin
to hate him and he will be little esteemed being poor, so that having
by this liberality injured many and benefited but few, he will feel
the first little disturbance and be endangered by every accident. If
he recognises this and wishes to change his system, he incurs at once
the charge of niggardliness; a prince, therefore, not being able to
exercise this virtue of liberality without risk if it is known, must
not, if he is prudent, object to be called miserly. In course of time
he will be thought more liberal, when it is seen that by his parsimony
his revenue is sufficient, that he can defend himself against those
who make war on him, and undertake enterprises without burdening his
people, so that he is really liberal to all those from whom he does not
take, who are infinite in number, and niggardly to all to whom he does
not give, who are few.

In our times we have seen nothing great done except by those who have
been esteemed niggardly; the others have all been ruined. Pope Julius
II., although he had made use of a reputation for liberality in order
to attain the papacy, did not seek to retain it afterwards, so that
he might be able to make war on the King of France, and he earned on
so many wars without imposing an extraordinary tax, because his extra
expenses were covered by the parsimony he had so long practised. The
present King of Spain, if he had been thought liberal, would not have
engaged in and won so many enterprises. For these reasons a prince must
care little for the reputation of being a miser, if he wishes to avoid
robbing his subjects, if he wishes to be able to defend himself, to not
become poor and contemptible, and not to be forced to become rapacious;
this vice of niggardliness is one of those vices which enable him to
reign. If it is said that Cæsar attained the empire through liberality,
and that many others have reached the highest positions through being
liberal or being thought so, I would reply that you are either a prince
already or else on the way to become one. In the first case, this
liberality is harmful; in the second, it is certainly necessary to be
considered liberal, and Cæsar was one of those who wished to attain the
mastery over Rome, but if after attaining it he had lived and had not
moderated his expenses, he would have destroyed that empire. And should
any one reply that there have been many princes, who have done great
things with their armies, who have been thought extremely liberal, I
would answer by saying that the prince may either spend his own wealth
and that of his subjects or the wealth of others. In the first case
he must be sparing, but in the second he must not neglect to be very
liberal. This liberality is very necessary to a prince who marches with
his armies, and lives by plunder, sacking and extorting, and is dealing
with the wealth of others, for without it he would not be followed
by his soldiers. And you may be very generous indeed with what is
not the property of yourself or your subjects, as were Cyrus, Cæsar,
and Alexander; for spending the wealth of others will not diminish
your reputation, but increase it, only spending your own resources
will injure you. There is nothing which destroys itself so much as
liberality, for by using it you lose the power of using it, and become
either poor and despicable, or, to escape poverty, rapacious and hated.
And of all things that a prince must guard against, the most important
are being despicable or hated, and liberality will lead you to one or
other of these conditions. It is, therefore, wiser to have the name
of a miser, which produces disgrace without hatred, than to incur of
necessity the name of being rapacious, which produces both disgrace and
hatred.

\chapter{Of Cruelty and Clemency, and Whether It Is Better to Be Loved or Feared}

Proceeding to the other qualities before named, I say that every prince
must desire to be considered merciful and not cruel. He must, however,
take care not to misuse this mercifulness. Cesare Borgia was considered
cruel, but his cruelty had settled the Romagna, united it, and brought
it peace and confidence. If this is considered a benefit, it will be
seen that he was really much more merciful than the Florentine people,
who, to avoid the name of cruelty, allowed Pistoia to be destroyed. A
prince, therefore, must not mind incurring the charge of cruelty for
the purpose of keeping his subjects united and confident; for, with a
very few examples, he will be more merciful than those who, from excess
of tenderness, allow disorders to arise, from whence spring murders
and rapine; for these as a rule injure the whole community, while the
executions carried out by the prince injure only one individual. And of
all princes, it is impossible for a new prince to escape the name of
cruel, new states being always full of dangers. Wherefore Virgil makes
Dido excuse the inhumanity of her rule by its being new, where she says:

    Res dura, et regni novitas me talia cogunt
    Moliri, et late fines custode tueri.

Nevertheless, he must be cautious in believing and acting, and must
not inspire fear of his own accord, and must proceed in a temperate
manner with prudence and humanity, so that too much confidence does
not render him incautious, and too much diffidence does not render him
intolerant. From this arises the question whether it is better to be
loved more than feared, or feared more than loved. The reply is, that
one ought to be both feared and loved, but as it is difficult for the
two to go together, it is much safer to be feared than loved, if one of
the two has to be wanting. For it may be said of men in general that
they are ungrateful, voluble, dissemblers, anxious to avoid danger,
and covetous of gain; as long as you benefit them, they are entirely
yours; they offer you their blood, their goods, their life, and their
children, as I have before said, when the necessity is remote; but
when it approaches, they revolt And the prince who has relied solely
on their words, without making other preparations, is ruined, for the
friendship which is gained by purchase and not through grandeur and
nobility of spirit is merited but is not secured, and at times is not
to be had. And men have less scruple in offending one who makes himself
loved than one who makes himself feared; for love is held by a chain of
obligation which, men being selfish, is broken whenever it serves their
purpose; but fear is maintained by a dread of punishment which never
fails. Still, a prince should make himself feared in such a way that if
he does not gain love, he at any rate avoids hatred; for fear, and the
absence of hatred may well go together, and will be always attained by
one who abstains from interfering with the property of his citizens and
subjects or with their women. And when he is obliged to take the life
of any one, to do so when there is a proper justification and manifest
reason for it; but above all he must abstain from taking the property
of others, for men forget more easily the death of their father than
the loss of their patrimony. Then also pretexts for seizing property
are never wanting, and one who begins to live by rapine will always
find some reason for taking the goods of others, whereas causes for
taking life are rarer and more quickly destroyed. But when the prince
is with his army and has a large number of soldiers under his control,
then it is extremely necessary that he should not mind being thought
cruel; for without, this reputation he could not keep an army united,
or disposed to any duty.

Among the noteworthy actions of Hannibal is numbered this, that
although he had an enormous army, composed of men of all nations
and fighting in foreign countries, there never arose any dissension
either among them or against the prince, either in good fortune or in
bad. This could not be due to anything but his inhuman cruelty, which
together with his infinite other virtues, made him always venerated
and terrible in the sight of his soldiers, and without it his other
virtues would not have sufficed to produce that effect. Thoughtless
writers admire on the one hand his actions, and on the other blame the
principal cause of them. And that it is true that his other virtues
would not have sufficed may be seen from the case of Scipio (very rare
not only in his own times, but in all times of which memory remains),
whose armies rebelled against him in Spain, which arose from nothing
but his excessive kindness, which allowed more license to the soldiers
than was consonant with military discipline. He was reproached with
this in the senate by Fabius Maximus, who called him a corrupter of the
Roman militia.

The Locri having been destroyed by one of Scipio's officers were not
revenged by him, nor was the insolence of that officer punished,
simply by reason of his easy nature; so much so, that some one wishing
to excuse him in the senate, said that there were many men who knew
rather how not to err, than how to correct the errors of others. This
disposition would in time have tarnished the fame and glory of Scipio
had he persevered in it under the empire, but living under the rule of
the senate this harmful quality was not only concealed but became a
glory to him. I conclude, therefore, with regard to being feared and
loved, that men love at their own free will, but fear at the will of
the prince, and that a wise prince must rely on what is in his power
and not on what is in the power of others, and he must only trouble
himself to avoid incurring hatred, as has been explained.

\chapter{In What Way Princes Must Keep Faith}

How laudable it is for a prince to keep good faith and live with
integrity, and not with astuteness, every one knows. Still the
experience of our times shows those princes to have done great things
who have had little regard for good faith, and have been able by
astuteness to confuse men's brains, and who have ultimately overcome
those who have, made loyalty their foundation. You must know, then,
that there are two methods of fighting, the one by law, the other by
force: the first method is that of men, the second of beasts; but as
the first method is often insufficient, one must have recourse to the
second. It is therefore necessary to know well how to use both the
beast and the man. This was covertly taught to princes by ancient
writers, who relate how Achilles and many others of those princes were
given to Chiron the centaur to be brought up, who kept them under his
discipline; this system of having for teacher one who was half beast
and half man is meant to indicate that a prince must know how to use
both natures, and that the one without the other is not durable. A
prince being thus obliged to know well how to act as a beast must
imitate the fox and the lion, for the lion cannot protect himself
from snares, and the fox cannot defend himself from wolves. One must
therefore be a fox to recognise snares, and a lion to frighten wolves.
Those that wish to be only lions do not understand this. Therefore,
a prudent ruler ought not to keep faith when by so doing it would be
against his interest, and when the reasons which made him bind himself
no longer exist. If men were all good, this precept would not be a
good one; but as they are bad, and would not observe their faith with
you, so you are not bound to keep faith with them. Nor are legitimate
grounds ever wanting to a prince to give colour to the non-fulfilment
of his promise. Of this one could furnish an infinite number of modern
examples, and show how many times peace has been broken, and how many
promises rendered worthless, by the faithlessness of princes, and those
that have been best able to imitate the fox have succeeded best. But
it is necessary to be able to disguise this character well, and to be
a great feigner and dissembler; and men are so simple and so ready to
obey present necessities, that one who deceives will always find those
who allow themselves to be deceived. I will only mention one modern
instance. Alexander VI. did nothing else but deceive men, he thought
of nothing else, and found the way to do it; no man was ever more able
to give assurances, or affirmed things with stronger oaths, and no man
observed them less; however, he always succeeded in his deceptions, as
he knew well this side of the world. It is not, therefore, necessary
for a prince to have all the above-named qualities, but it is very
necessary to seem to have them. I would even be bold to say that to
possess them and to always observe them is dangerous, but to appear
to possess them is useful. Thus it is well to seem pious, faithful,
humane, religious, sincere, and also to be so; but you must have the
mind so watchful that when it is needful to be otherwise you may be
able to change to the opposite qualities. And it must be understood
that a prince, and especially a new prince, cannot observe all those
things which are considered good in men, being often obliged, in order
to maintain the state, to act against faith, against charity, against
humanity, and against religion. And, therefore, he must have a mind
disposed to adapt itself according to the wind, and as the variations
of fortune dictate, and, as I said before, not deviate from what is
good, if possible, but be able to do evil if necessitated. A prince
must take great care that nothing goes out of his mouth which is not
full of the above-named five qualities, and, to see and hear him, he
should seem to be all faith, all integrity, all humanity, and all,
religion. And nothing is more necessary than to seem to have this last
quality, for men in general judge more by the eyes than by the hands,
for every one can see, but very few have to feel. Everybody sees what
you appear to be, few feel what you are, and those few will not dare
to oppose themselves to the many, who have the majesty of the state to
defend them; and in the actions of men, and especially of princes, from
which there is no appeal, the end is everything.

Let a prince therefore aim at living and maintaining state the state,
the means will always be judged honourable and praised by every one,
for the vulgar is always taken by appearances and the result of things;
and the world consists only of the vulgar, and the few find a place
when the many have nothing to rest upon. A certain prince of the
present time, whom it is well not to name, never does anything but
preach peace and good faith, but he is really a great enemy to both,
and either of them, had he observed them, would have lost him both
state and reputation on many occasions.

\chapter{That We Must Avoid Being Despised and Hated}

But as I have now spoken of the most important of the qualities
in question, I will now deal briefly with the rest on the general
principle, that the prince must, as already stated, avoid those things
which will make him hated or despised; and whenever he succeeds in
this, he will have done his part, and will find no danger in other
vices.

He will chiefly become hated, as I said, by being rapacious, and
usurping the property and women of his subjects, which he must abstain
from doing, and whenever one does not attack the property or honour
of the generality of men, they will live contented; and one will only
have to combat the ambition of a few, who can be easily held in check
in many ways. He is rendered despicable by being thought changeable,
frivolous, effeminate, timid, and irresolute; which a prince must
guard against as a rock of danger, and manage so that his actions
show grandeur, high courage, seriousness, and strength; and as to the
government of his subjects, let his sentence be irrevocable, and let
him adhere to his decisions so that no one may think of deceiving him
or making him change. The prince who creates such an opinion of himself
gets a great reputation, and it is very difficult to conspire against
one who has a great reputation, and he will not easily be attacked, so
long as it is known that he is esteemed and reverenced by his subjects.
For a prince must have two kinds of fear: one internal as regards his
subjects, one external as regards foreign powers. From the latter he
can defend himself with good arms and good friends, and he will always
have good friends if he has good arms; and internal matters will always
remain quiet, if they are not perturbed by conspiracy; and even if
external powers sought to foment one, if he has ruled and lived as I
have described, he will always if he stands firm be able to sustain
every shock, as I have shown that Nabis the Spartan did. But with
regard to the subjects, if not acted on from outside, it is still to be
feared lest they conspire in secret, from which the prince may guard
himself well by avoiding hatred and contempt, and keeping the people
satisfied with him, which it is necessary to accomplish, as has been
related at length. And one of the most potent remedies that a prince
has against conspiracies, is that of not being hated or despised by
the mass of the people; for whoever conspires always believes that
he will satisfy the people by the death of their prince; but if he
thought to offend them by doing this, he would fear to engage in such
an undertaking, for the difficulties that conspirators have to meet are
infinite. Experience shows that there have been very many conspiracies,
but few have turned out well, for whoever conspires cannot act alone,
and cannot find companions except among those who are discontented;
and as soon as you have disclosed your intention to a malcontent, you
give him the means of satisfying himself, for by revealing it he can
hope to secure everything he wants; to such an extent that seeing
a certain gain by doing this, and seeing on the other hand only a
doubtful one and full of danger, he must either be a rare friend to
you or else a very bitter enemy to the prince if he keeps faith with
you. And to reduce the matter to narrow limits, I say, that on the side
of the conspirator there is nothing but fear, jealousy, suspicion,
and dread of punishment which frightens him; and on the side of the
prince there is the majesty of government, the laws, the protection
of friends and of the state which guard him. When to these things are
added the goodwill of the people, it is impossible that any one should
have the temerity to conspire. For whereas generally a conspirator
has to fear before the execution of his plot, in this case he must
also fear afterwards, having the people for an enemy, when his crime
is accomplished, and thus not being able to hope for any refuge.
Numberless instances might be given of this, but I will content myself
with one which took place within the memory of our fathers. Messer
Annibale Bentivogli, Prince of Bologna, ancestor of the present Messer
Annibale, was killed by the Canneschi, who conspired against him. He
left no relations but Messer Giovanni, who was then an infant, but
after the murder the people rose up and killed all the Canneschi. This
arose from the popular goodwill that the house of Bentivogli enjoyed at
that time in Bologna, which was so great that, as there was nobody left
after the death of Annibale who could govern the state, the Bolognese
hearing that there was one of the Bentivogli family in Florence, who
had till then been thought the son of a blacksmith, came to fetch him
and gave him the government of the city, and it was governed by him
until Messer Giovanni was old enough to assume the government.

I conclude, therefore, that a prince need trouble little about
conspiracies when the people are well disposed, but when they
are hostile and hold him in hatred, then he must fear everything
and everybody. Well-ordered states and wise princes have studied
diligently not to drive the nobles to desperation, and to satisfy the
populace and keep it contented, for this is one of the most important
matters that a prince has to deal with. Among the kingdoms that are
well ordered and governed in our time is France, and there we find
numberless good institutions on which depend the liberty and security
of the king; of these the chief is the parliament and its authority,
because he who established that kingdom, knowing the ambition and
insolence of the great nobles, and deeming it necessary to have a bit
in their mouths to check them; and knowing on the other hand the hatred
of the mass of the people to the great, based on fear, and wishing to
secure them, did not wish to make this the special care of the king,
to relieve him of the dissatisfaction that he might incur among the
nobles by favouring the people, and among the people by favouring the
nobles. He therefore established a third judge that, without direct
charge of the king, kept in check the great and favoured the lesser
people. Nor could any better or more prudent measure have been adopted,
nor better precaution for the safety of the king and the kingdom. From
which another notable rule can be drawn, that princes should let the
carrying out of unpopular duties devolve on others, and bestow favours
themselves. I conclude again by saying that a prince must esteem his
nobles, but not make himself hated by the populace. It may perhaps seem
to some, that considering the life and death of many Roman emperors
that they are instances contrary to my opinion, finding that some who
lived always nobly and showed great strength of character, nevertheless
lost the empire, or were killed by their subjects who conspired against
them. Wishing to answer these objections, I will discuss the qualities
of some emperors, showing the cause of their ruin not to be at variance
with what I have stated, and I will also partly consider the things to
be noted by whoever reads the deeds of these times. I will content
myself with taking all those emperors who succeeded to the empire from
Marcus the philosopher to Maximinus; these were Marcus, Commodus his
son, Pertinax, Heliogabalus, Alexander, and Maximinus. And the first
thing to note is, that whereas other princes have only to contend
against the ambition of the great and the insolence of the people, the
Roman emperors had a third difficulty, that of having to support the
cruelty and avarice of the soldiers, which was such a difficulty that
it was the cause of the ruin of many, it being difficult to satisfy
both the soldiers and the people. For the people love tranquillity,
and therefore like princes who are pacific, but the soldiers prefer
a prince of military spirit, who is insolent, cruel, and rapacious.
They wish him to exercise these qualities on the people so that they
may get double pay and give vent to their avarice and cruelty. Thus it
came about that those emperors who, by nature or art, had not such a
reputation as could keep both parties in check, invariably were ruined,
and the greater number of them who were raised to the empire being
new men, knowing the difficulties of these two opposite dispositions,
confined themselves to satisfying the soldiers, and thought little of
injuring the people. This choice was necessary, princes not being able
to avoid being hated by some one. They must first try not to be hated
by the mass of the people; if they cannot accomplish this they must
use every means to escape the hatred of the most powerful parties. And
therefore these emperors, who being new men had need of extraordinary
favours, adhered to the soldiers more willingly than to the people;
whether this, however, was of use to them or not, depended on whether
the prince knew how to maintain his reputation with them.

From these causes it resulted that Marcus, Pertinax, and Alexander,
being all of modest life, lovers of justice, enemies of cruelty,
humane and benign, had all a sad ending except Marcus. Marcus alone
lived and died in honour, because he succeeded to the empire by
hereditary right and did not owe it either to the soldiers or to the
people; besides which, possessing many virtues which made him revered,
he kept both parties in their place as long as he lived and was never
either hated or despised. But Pertinax was created emperor against the
will of the soldiers, who being accustomed to live licentiously under
Commodus, could not put up with the honest life to which Pertinax
wished to limit them, so that having made himself hated, and to this
contempt being added because he was old, he was ruined at the very
beginning of his administration. Whence it may be seen that hatred
is gained as much by good works as by evil, and therefore, as I said
before, a prince who wishes to maintain the state is often forced to
do evil, for when that party, whether populace, soldiery, or nobles,
whichever it be that you consider necessary to you for keeping your
position, is corrupt, you must follow its humour and satisfy it, and
in that case good works will be inimical to you. But let us come to
Alexander, who was of such goodness, that among other things for which
he is praised, it is said that in the fourteen years that he reigned
no one was put to death by him without a fair trial. Nevertheless,
being considered effeminate, and a man who allowed himself to be
ruled by his mother, and having thus fallen into contempt, the army
conspired against him and killed him. Looking, on the other hand,
at the qualities of Commodus, Severus, Antoninus, extremely cruel
and rapacious; to satisfy the soldiers there was no injury which
they would not inflict on the people, and all except Severus ended
badly. Severus, however, had such abilities that by maintaining the
soldiers friendly to him, he was able to reign happily, although he
oppressed the people, for his virtues made him so admirable in the
sight both of the soldiers and the people that the latter were, as it
were, astonished and stupefied, while the former were respectful and
contented. As the deeds of this ruler were great for a new prince, I
will briefly show how well he could use the qualities of the fox and
the lion, whose natures, as I said before, it is necessary for a prince
to imitate. Knowing the sloth of the Emperor Julian, Severus, who was
leader of the army in Slavonia, persuaded the troops that it would be
well to go to Rome to avenge the death of Pertinax, who had been slain
by the Imperial guard, and under this pretext, without revealing his
aspirations to the throne, marched with his army to Rome and was in
Italy before his design was known. On his arrival in Rome the senate
elected him emperor through fear, and Julian died. There remained
after this beginning two difficulties to be faced by Severus before
he could obtain the whole control of the empire: one in Asia, where
Nigrinus, head of the Asiatic armies, had declared himself emperor; the
other in the west from Albinus, who also aspired to the empire. And
as he judged it dangerous to show himself hostile to both, he decided
to attack Nigrinus and deceive Albinus, to whom he wrote that having
been elected emperor by the senate he wished to share that dignity
with him; he sent him the title of Cæsar and, by deliberation of the
senate, he was declared his colleague; all of which was accepted as
true by Albinus. But when Severus had defeated and killed Nigrinus,
and pacified things in the East, he returned to Rome and charged
Albinus in the senate with having, unmindful of the benefits received
from him, traitorously sought to assassinate him, and stated that he
was therefore obliged to go and punish his ingratitude. He then went
to France to meet him, and there deprived him of both his position
and his life. Whoever examines in detail the actions of Severus, will
find him to have been a very ferocious lion and an extremely astute
fox, and will see him to have been feared and respected by all and not
hated by the army; and will not be surprised that he, a new man, should
have been able to hold the empire so well, since his great reputation
defended him always from that hatred that his rapacity might have
produced in the people. But Antoninus his son was also a man of great
ability, and possessed qualities that rendered him admirable in the
sight of the people and also made him popular with the soldiers, for
he was a military man, capable of enduring the most extreme hardships,
disdainful of delicate food, and every other luxury, which made him
loved by all the armies. However, his ferocity and cruelty were so
great and unheard of, through his having, after executing many private
individuals, caused a large part of the population of Rome and all that
of Alexandria to be killed, that he became hated by all the world and
began to be feared by those about him to such an extent that he was
finally killed by a centurion in the midst of his army. Whence it is to
be noted that this kind of death, which proceeds from the deliberate
action of a determined man, cannot be avoided by princes, since any one
who does not fear death himself can inflict it, but a prince need not
fear much on this account, as such actions are extremely rare. He must
only guard against committing any grave injury to any one he makes use
of, or has about him for his service, like Antoninus had done, having
caused the death with contumely of the brother of that centurion, and
also threatened him every day, although he still retained him in his
bodyguard, which was a foolish and dangerous thing to do, as the fact
proved. But let us come to Commodus, who might easily have kept the
empire, having succeeded to it by heredity, being the son of Marcus,
and it would have sufficed for him to follow in the steps of his father
to have satisfied both the people and the soldiers. But being of a
cruel and bestial disposition, in order to be able to exercise his
rapacity on the people, he sought to amuse the soldiers and render
them licentious; on the other hand, by not maintaining his dignity,
by often descending into the theatre to fight with gladiators and
committing other contemptible actions, little worthy of the imperial
dignity, he became despicable in the eyes of the soldiers, and being
hated on the one hand and despised on the other, he was conspired
against and killed. There remains to be described the character of
Maximinus. He was an extremely warlike man, and as the armies were
annoyed with the effeminacy of Alexander, which we have already spoken
of, he was after the death of the latter elected emperor. He did not
enjoy it for long, as two things made him hated and despised: the
one his base origin, as he had been a shepherd in Thrace, which was
generally known and caused great disdain on all sides; the other,
because he had at the commencement of his rule deferred going to Rome
to take possession of the Imperial seat, and had obtained a reputation
for great cruelty, having through his prefects in Rome and other parts
of the empire committed many acts of cruelty. The whole world being
thus moved by indignation for the baseness of his blood, and also by
the hatred caused by fear of his ferocity, he was conspired against
first by Africa and afterwards by the senate and all the people of Rome
and Italy. His own army also joined them, for besieging Aquileia and
finding it difficult to take, they became enraged at his cruelty, and
seeing that he had so many enemies, they feared him less and put him to
death. I will not speak of Heliogabalus, of Macrinus, or Julian, who
being entirely contemptible were immediately suppressed, but I will
come to the conclusion of this discourse by saying that the princes of
our time have less difficulty than these of being obliged to satisfy in
an extraordinary degree their soldiers in their states; for although
they must have a certain consideration for them, yet it is soon
settled, for none of these princes have armies that are inextricably
bound up with the administration of the government and the rule of
their provinces as were the armies of the Roman empire; and therefore
if it was then necessary to satisfy the soldiers rather than the
people, it was because the soldiers could do more than the people; now,
it is more necessary to all princes, except the Turk and the Soldan,
to satisfy the people than the soldiers, for the people can do more
than the soldiers. I except the Turk, because he always keeps about
him twelve thousand infantry and fifteen thousand cavalry, on which
depend the security and strength of his kingdom; and it is necessary
for him to postpone every other consideration of the people to keep
them friendly. It is the same with the kingdom of the Soldan, which
being entirely in the hands of the soldiers, he is bound to keep their
friendship regardless of the people. And it is to be noted that this
state of the Soldan is different from that of all other princes, being
similar to the Christian pontificate, which cannot be called either
a hereditary kingdom or a new one, for the sons of the dead prince
are not his heirs, but he who is elected to that position by those
who have authority. And as this order is ancient it cannot be called
a new kingdom, there being none of these difficulties which exist in
new ones; as although the prince is new, the rules of that state are
old and arranged to receive him as if he were their hereditary lord.
But returning to our matter, I say that whoever studies the preceding
argument will see that either hatred or contempt were the causes of
the ruin of the emperors named, and will also observe how it came about
that, some of them acting in one way and some in another, in both ways
there were some who had a fortunate and others an unfortunate ending.
As Pertinax and Alexander were both new rulers, it was useless and
injurious for them to try and imitate Marcus, who was a hereditary
prince; and similarly with Caracalla, Commodus, and Maximinus it was
pernicious for them to imitate Severus, as they had not sufficient
ability to follow in his footsteps. Thus a new prince cannot imitate
the actions of Marcus, in his dominions, nor is it necessary for him to
imitate those of Severus; but he must take from Severus those portions
that are necessary to found his state, and from Marcus those that are
useful and glorious for conserving a state that is already established
and secure.

\chapter{Whether Fortresses and Other Things which Princes Often Make Are Useful or Injurious}

Some princes, in order to securely hold their possessions, have
disarmed their subjects, some others have kept their subject lands
divided into parts, others have fomented enmities against themselves,
others have endeavoured to win over those whom they suspected at the
commencement of their rule: some have constructed fortresses, others
have ruined and destroyed them. And although one cannot pronounce a
definite judgment as to these things without going into the particulars
of the state to which such a deliberation is to be applied, still I
will speak in such a broad way as the matter will permit of.

A new prince has never been known to disarm his subjects, on the
contrary, when he has found them disarmed he has always armed them, for
by arming them these arms become your own, those that you suspected
become faithful and those that were faithful remain so, and from being
merely subjects become your partisans. And since all the subjects
cannot be armed, when you benefit those that you arm, you can deal more
safely with the others; and this different treatment that they notice
renders your men more obliged to you, the others will excuse you,
judging that those have necessarily greater merit who have greater
danger and heavier duties. But when you disarm them, you commence to
offend them and show that you distrust them either through cowardice or
lack of confidence, and both of these opinions generate hatred against
you. And as you cannot remain unarmed, you are obliged to resort to a
mercenary militia, of which we have already stated the value; and even
if it were good it cannot be sufficient in number to defend you against
powerful enemies and suspected subjects. But, as I have said, a new
prince in a new dominion always has his subjects armed. History is full
of such examples. But when a prince acquires a new state as an addition
to his old one, then it is necessary to disarm that state, except those
who in acquiring it have sided with you; and even these one must, when
time and opportunity serve, render weak and effeminate, and arrange
things so that all the arms of the new state are in the hands of your
own soldiers who in your old state live near you.

Our forefathers and those who were esteemed wise used to say that
it was necessary to hold Pistoia by means of factious and Pisa with
fortresses, and for this purpose they fomented differences among their
subjects in some town in order to possess it more easily. This, in
those days when Italy was fairly divided, was doubtless well done, out
does not seem to me to be a good precept for the present time, for I do
not believe that the divisions thus created ever do any good; on the
contrary it is certain that when the enemy approaches the cities thus
divided will be at once lost, for the weaker faction will always side
with the enemy and the other will not be able to stand. The Venetians,
actuated, I believe, by the aforesaid motives, cherished the Guelf
and Ghibelline factions in the cities subject to them, and although
they never allowed them to come to bloodshed, they yet encouraged
these differences among them, so that the citizens, being occupied in
their own quarrels, might not act against them. This, however, did
not avail them anything, as was seen when, after the defeat of Vaila,
a part of those subjects immediately took courage and took from them
the whole state. Such methods, besides, argue weakness in a prince,
for in a strong government such dissensions will never be permitted.
They are profitable only in time of peace, as by means of them it is
easy to manage one's subjects, but when it comes to war, the fallacy
of such a policy is at once shown. Without doubt princes become great
when they overcome difficulties and opposition, and therefore fortune,
especially when it wants to render a new prince great, who has greater
need of gaining a great reputation than a hereditary prince, raises up
enemies and compels him to undertake wars against them, so that he may
have cause to overcome them, and thus raise himself higher by means
of that ladder which his enemies have brought him. There are many who
think therefore that a wise prince ought, when he has the chance, to
foment astutely some enmity, so that by suppressing it he will augment
his greatness. Princes, and especially new ones, have found more faith
and more usefulness in those men, whom at the beginning of their power
they regarded with suspicion, than in those they at first confided
in. Pandolfo Petrucci, Prince of Siena, governed his state more by
those whom he suspected than by others. But of this we cannot speak at
large, as it varies according to the subject; I will merely say that
these men who at the beginning of a new government were enemies, if
they are of a kind to need support to maintain their position, can be
very easily gained by the prince, and they are the more compelled to
serve him faithfully as they know they must by their deeds cancel the
bad opinion previously held of them, and thus the prince will always
derive greater help from them than from those who, serving him with
greater security, neglect his interest? And as the matter requires it,
I will not omit to remind a prince who has newly taken a state with the
secret help of its inhabitants, that he must consider well the motives
that have induced those who have favoured him to do so, and if it is
not natural affection for him, but only because they were not contented
with the state as it was, he will have great trouble and difficulty in
maintaining their friendship, because it will be impossible for him to
content them. And on well examining the cause of this in the examples
drawn from ancient and modern times it will be seen that it is much
easier to gain the friendship, of those men who were contented with
the previous condition and were therefore at first enemies, than that
of those who not being contented, became his friends and helped him to
occupy it. It has been the custom of princes in order to be able to
hold securely their state, to erect fortresses, as a bridle and bit
to those who have designs against them, and in order to have a secure
refuge against a sudden assault. I approve this method, because it was
anciently used. Nevertheless, Messer Niccolo Vitelli has been seen in
our own time to destroy two fortresses in Città di Castello in order
to keep that state. Guid' Ubaldo, Duke of Urbino, on returning to his
dominions from which he had been driven by Cesare Borgia, razed to
their foundations all the fortresses of that province, and considered
that without them it would be more difficult for him to lose again the
state. The Bentivogli, in returning to Bologna, used similar measures.
Therefore fortresses may or may not be useful according to the times;
if they do good in one way, they do harm in another.

The question may be discussed thus: a prince who fears his own people
more than foreigners ought to build fortresses, but he who has greater
fear of foreigners than of his own people ought to do without them. The
castle of Milan built by Francesco Sforza has given and will give more
trouble to the house of Sforza than any other disorder in that state.
Therefore the best fortress is to be found in the love of the people,
for although you may have fortresses they will not save you if you are
hated by the people. When once the people have taken arms against you,
there will never be lacking foreigners to assist them. In our times
we do not see that they have profited any ruler, except the Countess
of Forli on the death of her consort Count Girolamo, for she was thus
enabled to escape the popular rising and await help from Milan and
recover the state; the circumstances being then such that no foreigner
could assist the people. But afterwards they were of little use to her
when Cesare Borgia attacked her and the people being hostile to her
allied themselves with the foreigner. So that then and before it would
have been safer for her not to be hated by the people than to have the
fortresses. Having considered these things I would therefore praise the
one who erects fortresses and the one who does not, and would blame any
one who, trusting in them, thinks little of being hated by his people.

\chapter{How a Prince Must Act in Order to Gain Reputation}

Nothing causes a prince to be so much esteemed as great enterprises
and setting a rare example. We have in our own day Ferdinand, King of
Aragon, at present King of Spain. He may almost be termed a new prince,
because from a weak king he has become for fame and glory the first
king in Christendom, and if you regard his actions you will find them
all very great and some of them extraordinary. At the beginning of
his reign he assailed Granada, and that enterprise was the foundation
of his state. At first he did it leisurely and without fear of being
interfered with; he kept the minds of the barons of Castile occupied in
this enterprise, so that thinking only of that war they did not think
of making innovations, and he thus acquired reputation and power over
them without their being aware of it. He was able with the money of
the Church and the people to maintain his armies, and by that long war
lay the foundations of his military power, which afterwards has made
him famous. Besides this, to be able to undertake greater enterprises,
and always under the pretext of religion, he had recourse to a pious
cruelty, driving out the Moors from his kingdom and despoiling them.
No more admirable or rare example can be found. He also attacked
under the same pretext Africa, undertook his Italian enterprise, and
has lately attacked France; so that he has continually contrived great
things, which have kept his subjects' minds uncertain and astonished,
and occupied in watching their result.

And these actions have arisen one out of the other, so that they have
left no time for men to settle down and act against him. It is also
very profitable for a prince to give some rare examples of himself in
the internal administration, like those related of Messer Bernabò of
Milan, when it happens that some one does something extraordinary,
either good or evil, in civil life, and to take a means of rewarding
or punishing him which will be much talked about. And above all a
prince must endeavour in every action to obtain fame for being great
and excellent. A prince is further esteemed when he is a true friend
or a true enemy, when, that is, he declares himself without reserve in
favour of some one against another.

This policy is always more useful than remaining neutral. For if two
neighbouring powers come to blows, they are either such that if one
wins, you will have to fear the victor, or else not. In either of these
two cases it will be better for you to declare yourself openly and make
war, because in the first case if you do not declare yourself, you will
fall a prey to the victor, to the pleasure and satisfaction of the one
who has been defeated, and you will have no reason nor anything to
defend you and nobody to receive you. For, whoever wins will not desire
friends whom he suspects and who do not help him when in trouble, and
whoever loses will not receive you as you did not take up arms to
assist his cause. Antiochus went to Greece, being sent by the Ætoli
to expel the Romans. He sent orators to the Achæi who were friends of
the Romans to encourage them to remain neutral, on the other hand the
Romans persuaded them to take up arms on their side. The matter was
brought before the council of the Achæi for deliberation, where the
ambassador of Antiochus sought to persuade them to remain neutral, to
which the Roman ambassador replied: "As to what is said that it is best
and most useful for your state not to meddle in our war, nothing is
further from the truth; for if you do not meddle in it you will become,
without any favour or any reputation, the prize of the victor." And it
will always happen that the one who is not your friend will want you
to remain neutral, and the one who is your friend will require you to
declare yourself by taking arms. Irresolute princes, to avoid present
dangers, usually follow the way of neutrality and are mostly ruined
by it. But when the prince declares himself frankly in favour of one
side, if the one to whom you adhere conquers, even if he is powerful
and you remain at his discretion, he is under an obligation to you and
friendship has been established, and men are never so dishonest as to
oppress you with such ingratitude.

Moreover, victories are never so prosperous that the victor does not
need to have some scruples, especially as to justice. But if he to whom
you adhere loses, you are sheltered by him, and so long as he can, he
will assist you; you become the companion of a fortune which may rise
again. In the second case, when those who fight are such that you have
nothing to fear from the victor, it is still more prudent on your part
to adhere to one; for you go to the ruin of one with the help of him
who ought to save him if he were wise, and if he conquers he rests
at your discretion, and it is impossible that he should not conquer
with your help. And here it should be noted that a prince ought never
to make common cause with one more powerful than himself to injure
another, unless necessity forces him to it, as before said; for if he
wins you rest at his discretion, and princes must avoid as much as
possible being at the discretion of others. The Venetians united with
France against the Duke of Milan, although they could have avoided that
union, and from it resulted their own ruin. But when one cannot avoid
it, as happened to the Florentines when the pope and Spain went with
their armies to attack Lombardy, then the prince ought to join for the
above reasons. Let no state believe that it can follow a secure policy,
rather let it think that all are doubtful. This is found in the nature
of things, that one never tries to avoid one difficulty without running
into another, but prudence consists in being able to know the nature of
the difficulties, and taking the least harmful as good. A prince must
also show himself a lover of merit, and honour those who excel in every
art. Moreover he must encourage his citizens to follow their callings
quietly, whether in commerce, or agriculture, or any other trade that
men follow, so that this one shall not refrain from improving his
possessions through fear that they may be taken from him, and that one
from starting a trade for fear of taxes; but he should offer rewards to
whoever does these things, and to whoever seeks in any way to improve
his city or state. Besides this, he ought, at convenient seasons of the
year, to keep the people occupied with festivals and spectacles; and
as every city is divided either into trades or into classes, he ought
to pay attention to all these things, mingle with them from time to
time, and give them an example of his humanity and magnificence, always
holding firm, however, the majesty of his dignity, which must never be
allowed to fan in anything whatever.

\chapter{Of the Secretaries of Princes}

The choice of a prince's ministers is a matter of no little importance;
they are either good or not according to the prudence of the prince.
The first impression that one gets of a ruler and of his brains is
from seeing the men that he has about him. When they are competent
and faithful one can always consider him wise, as he has been able to
recognise their ability and keep them faithful. But when they are the
reverse, one can always form an unfavourable opinion of him, because
the first mistake that he makes is in making this choice. There was
nobody who knew Messer Antonio da Venafro as the minister of Pandolfo
Petrucci, Prince of Siena, who did not consider Pandolfo to be a very
prudent man, having him for his minister. There are three different
kinds of brains, the one understands things unassisted, the other
understands things when shown by others, the third understands neither
alone nor with the explanations of others. The first kind is most
excellent, the second also excellent, but the third useless. It is
therefore evident that if Pandolfo was not of the first kind, he was
at any rate of the second. For every time that one has the judgment to
know the good and evil that any one does or says, even if he has no
invention, yet he recognises the bad and good works or his minister
and corrects the one and supports the other; and the minister cannot
hope to deceive him and therefore remains good. For a prince to be
able to know a minister there is this method which never fails. When
you see the minister think more of himself than of you, and in all his
actions seek his own profit, such a man will never be a good minister,
and you can never rely on him; for whoever has in hand the state of
another must never think of himself but of the prince, and not call
to mind anything but what relates to him. And, on the other hand, the
prince, in order to retain his fidelity ought to think of his minister,
honouring and enriching him, doing him kindnesses, and conferring on
him honours and giving him responsible tasks, so that the great honours
and riches bestowed on him cause him not to desire other honours and
riches, and the tasks he has to fulfil make him fearful of changes,
knowing that he could not execute them without the prince. When princes
and their ministers stand in this relation to each other, they can
rely the one upon the other; when it is otherwise, the end is always
injurious either for one or the other of them.

\chapter{How Flatterers Must Be Shunned}

I must not omit an important subject, and a mistake which princes
can with difficulty avoid, if they are not very prudent, or if they
do not make a good choice. And this is with regard to flatterers, of
which courts are full, because men take such pleasure in their own
things and deceive themselves about them that they can with difficulty
guard against this plague; and by wishing to guard against it they run
the risk of becoming contemptible. Because there is no other way of
guarding one's self against flattery than by letting men understand
that they will not offend you by speaking the truth; but when every one
can tell you the truth, you lose their respect. A prudent prince must
therefore take a third course, by choosing in his state wise men, and
giving these alone full liberty to speak the truth to him, but only of
those things that he asks and of nothing else; but he must ask them
about everything and hear their opinion, and afterwards deliberate by
himself in his own way, and in these councils and with each of these
men comport himself so that every one may see that the more freely he
speaks, the more he will be acceptable. Outside these he should listen
to no one, go about the matter deliberately, and be determined in his
decisions. Whoever acts otherwise either acts precipitately through
flattery or else changes often through the variety of opinions, from
which it happens that he is little esteemed. I will give a modern
instance of this. Pre' Luca, a follower of Maximilian, the present
emperor, speaking of his majesty said that he never took counsel with
anybody, and yet that he never did anything as he wished; this arose
from his following the contrary method to the aforesaid. As the emperor
is a secret man he does not communicate his designs to any one or
take any one's advice, but as on putting them into effect they begin
to be known and discovered, they begin to be opposed by those he has
about him, and he is easily diverted from his purpose. Hence it comes
to pass that what he does one day he undoes the next, no one ever
understands what he wishes or intends to do, and no reliance is to be
placed on his deliberations. A prince, therefore, ought always to take
counsel, but only when he wishes, not when others wish; on the contrary
he ought to discourage absolutely attempts to advise him unless he
asks it, but he ought to be a great asker, and a patient hearer of
the truth about those things which he has inquired of; indeed, if he
finds that any one has scruples in telling him the truth he should be
angry. And since some think that a prince who gains the reputation
of being prudent is so considered, not by his nature but by the good
councillors he has about him, they are undoubtedly deceived. It is an
infallible rule that a prince who is not wise himself cannot be well
advised, unless by chance he left himself entirely in the hands of one
man who ruled him in everything, and happened to be a very prudent
man. In this case he may doubtless be well governed, but it would not
last long, for that governor would in a short time deprive him of the
state; but by taking counsel with many, a prince who is not wise will
never have united councils and will not be able to unite them for
himself. The councillors will all think of their own interests, and he
will be unable either to correct or to understand them. And it cannot
be otherwise, for men will always be false to you unless they are
compelled by necessity to be true.

Therefore it must be concluded that wise counsels, from whoever they
come, must necessarily be due to the prudence of the prince, and not
the prudence of the prince to the good counsels received.

\chapter{Why the Princes of Italy Have Lost Their States}

The before-mentioned things, if prudently observed, make a new prince
seem ancient, and render him at once more secure and firmer in the
state than if he had been established there of old. For a new prince is
much more observed in his actions than a hereditary one, and when these
are recognised as virtuous, he gains men more and they are more bound
to him than if he were of the ancient blood. For men are much more
taken by present than by past things, and when they find themselves
well off in the present, they enjoy it and seek nothing more; on the
contrary, they will do all they can to defend him, so long as the
prince is not in other things wanting to himself. And thus he will
have the double glory of having founded a new realm and adorned it and
fortified it with good laws, good arms, good friends and good examples;
as he will have double shame who is born a prince and through want of
prudence has lost it.

And if one considers those rulers who have lost their position in
Italy in our days, such as the King of Naples, the Dukè of Milan and
others, one will find in them first a common defect as to their arms,
for the reasons discussed at length, then we observe that some of
them either had the people hostile to them, or that if the people were
friendly they were not able to make sure of the nobility, for without
these defects, states are not lost that have enough strength to be able
to keep an army in the field. Philip of Macedon, not the father of
Alexander the Great, but the one who was conquered by Titus Quinteus,
did not possess a great state compared to the greatness of Rome and
Greece which assailed him, but being a military man and one who knew
how to divert the people and make sure of the great, he was able to
sustain the war against them for many years; and if at length he lost
his power over several cities, he was still able to keep his kingdom.
Therefore, those of our princes who had held their possessions for many
years must not accuse fortune for having lost them, but rather their
own negligence; for having never in quiet times considered that things
might change (as it is a common fault of men not to reckon on storms,
in fair weather) when adverse times came, they only thought of fleeing
from them, instead of defending themselves; and hoped that the people,
enraged by the insolence of the conquerors, would recall them. This
measure, when others are wanting, is good; but it is very bad to have
neglected the other remedies for that one, for nobody would desire to
fall because he believed that he would then find some one to pick him
up. This may or may not take place, and if it does, it is not with
safety to you, as that defence is known to be cowardly and not to be
depended on; and only those defences are good, certain and durable,
which depend only on yourself and your own ability.

\chapter{How Much Fortune Can Do in Human Affairs and How It May Be Opposed}

It is not unknown to me how many have been and are of opinion that
worldly events are so governed by fortune and by God, that men cannot
by their prudence change them, and that on the contrary there is no
remedy whatever, and for this they may judge it to be useless to toil
much about them, but let things be ruled by chance. This opinion has
been more believed in in our day, from the great changes that have been
seen, and are daily seen, beyond every human conjecture.

When I think about them at times, I am partly inclined to share
this opinion. Nevertheless, that our freewill may not be altogether
extinguished, I think it may be true that fortune is the ruler of half
our actions, but that she allows the other half or a little less to be
governed by us. I would compare her to an impetuous river that, when
turbulent, inundates the plains, ruins trees and buildings, removes
earth from this side and places it on the other; every one flies before
it, and everything yields to its fury without being able to oppose it;
and yet though it is of such a kind, still when it is quiet, men can
make provision against it by dams and banks, so that when it rises
it will either go into a canal or its rush will not be so wild and
dangerous. It happens similarly with fortune, which shows her power
where no measures have been taken to resist her, and turns her fury
where she knows that no dams or barriers have been made to hold her.
And if you regard Italy, which has been the seat of these changes, and
who has given the impulse to them, you will see her to be a country
without dams or barriers of any kind. If she had been protected by
proper measures, like Germany, Spain, and France, this inundation
would not have caused the great changes that it has, or would not have
happened at all. This must suffice as regards opposition to fortune
in general. But limiting myself more to particular cases, I would
point out how one sees a certain prince to-day fortunate and to-morrow
ruined, without seeing that he has changed in character or otherwise.
I believe this arises in the first place from the causes that we have
already discussed at length; that is to say, because the prince who
bases himself entirely on fortune is ruined when fortune varies. I
also believe that he is happy whose mode of proceeding accords with
the needs of the times, and similarly he is unfortunate whose mode of
proceeding is opposed to the times. For one sees that men in those
things which lead them to the aim that each one has in view, namely,
glory and riches, proceed in various ways; one with circumspection,
another with impetuosity, one by violence, another by cunning, one with
patience, another with the reverse; and each by these diverse ways
may arrive at his aim. One sees also two cautious men, one of whom
succeeds in his designs, and the other not, and in the same way two men
succeed equally by different methods, one being cautious, the other
impetuous, which arises only from the nature of the times, which does
or does not conform to their method of proceeding. From this results,
as I have said, that two men, acting differently, attain the same
effect, and of two others acting in the same way, one arrives at his
good and not the other. From this depend also the changes in fortune,
for if it happens that time and circumstances are favourable to one
who acts with caution and prudence he will be successful, but if time
and circumstances change he will be ruined, because he does not change
his mode of proceeding. No man is found able to adapt himself to this,
either because he cannot deviate from that to which his nature disposes
him, or else because having always prospered by walking in one path,
he cannot persuade himself that it is well to leave it; and therefore
the cautious man, when it is time to act suddenly, does not know how
to do so and is consequently ruined; for if one could change one's
nature with time and circumstances, fortune would never change. Pope
Julius II. acted impetuously in everything he did and found the times
and conditions so in conformity with that mode of proceeding, that he
always obtained a good result. Consider the first war that he made
against Bologna while Messer Giovanni Bentivogli was still living. The
Venetians were not pleased with it, the King of Spain and likewise
France had objections to this enterprise, notwithstanding which with
his fierce and impetuous disposition he engaged personally in the
expedition. This move caused both Spain and the Venetians to halt and
hesitate, the latter through fear, the former through the desire to
regain the entire kingdom of Naples. On the other hand, he engaged
with him the King of France, because seeing him make this move and
desiring his friendship in order to put down the Venetians, that king
judged that he could not refuse him his troops without manifest injury.
Thus Julius by his impetuous move achieved what no other pontiff with
the utmost human prudence would have succeeded in doing, because, if
he had waited till all arrangements had been made and everything
settled before leaving Rome, as any other pontiff would have done, it
would never have taken place. For the king of France would have found
a thousand excuses, and the others would have inspired him with a
thousand fears. I will omit his other actions, which were all of this
kind and which all succeeded well, and the shortness of his life did
not suffer him to experience the contrary, for had times succeeded
in which it was necessary to act with caution, his ruin would have
resulted, for he would never have deviated from these methods to which
his nature disposed him. I conclude then that fortune varying and men
remaining fixed in their ways, they are successful so long as these
ways conform to each other, but when they are opposed to each other
then they are unsuccessful. I certainly think that it is better to be
impetuous than cautious, for fortune is a woman, and it is necessary,
if you wish to master her, to conquer her by force; and it can be seen
that she lets herself be overcome by these rather than by those who
proceed coldly. And therefore, like a woman, she is a friend to the
young, because they are less cautious, fiercer, and master her with
greater audacity.

\chapter{Exhortation to Liberate Italy from the Barbarians}

Having now considered all the things we have spoken of, and thought
within myself whether at present the time was not propitious in Italy
for a new prince, and if there was not a state of things which offered
an opportunity to a prudent and capable man to introduce a new system
that would do honour to himself and good to the mass of the people, it
seems to me that so many things concur to favour a new ruler that I do
not know of any time more fitting for such an enterprise. And if, as
I said, it was necessary in order that the power of Moses should be
displayed that the people of Israel should be slaves in Egypt, and to
give scope for the greatness and courage of Cyrus that the Persians
should be oppressed by the Medes, and to illustrate the pre-eminence of
Theseus that the Athenians should be dispersed, so at the present time,
in order that the might of an Italian genius might be recognised, it
was necessary that Italy should be reduced to her present condition,
and that she should be more enslaved than the Hebrews, more oppressed
than the Persians, and more scattered than the Athenians; without a
head, without order, beaten, despoiled, lacerated, and overrun, and
that she should have suffered ruin of every kind. And although before
now a spirit has been shown by some which gave hope that he might be
appointed by God for her redemption, yet at the highest summit of his
career he was thrown aside by fortune, so that now, almost lifeless,
she awaits one who may heal her wounds and put a stop to the rapine and
pillaging of Lombardy, to the rapacity and extortion in the kingdom
and in Tuscany, and cure her of those sores which have long been
festering. Behold how she prays God to send some one to redeem her from
this barbarous cruelty and insolence. Behold her ready and willing to
follow any standard if only there be some one to raise it. There is
nothing now she can hope for but that your illustrious house may place
itself at the head of this redemption, being by its power and fortune
so exalted, and being favoured by God and the Church, whose leadership
it now occupies. Nor will this be very difficult to you, if you call to
mind the actions and lives of the men I have named. And although those
men were rare and marvellous, they were none the less men, and had each
of them less occasion than the present, for their enterprise was not
juster than this, nor easier, nor was God more their friend than He is
yours. Here is a just cause; for that war is just which is necessary;
and those arms are merciful where no hope exists save in them. Here
is the greatest willingness, nor can there be great difficulty where
there is great willingness, provided that the measures are adopted of
those whom I have set before you as examples. Besides this, unexampled
wonders have been seen here performed by God, the sea has been opened,
a cloud has shown you the road, the rock has given forth water, manna
has rained, and everything has contributed to your greatness, the
remainder must be done by you. God will not do everything, in order
not to deprive us of freewill and the portion of the glory that falls
to our lot It is no marvel that none of the before-mentioned Italians
have done that which it is to be hoped your illustrious house may do;
and if in so many revolutions in Italy and so many warlike operations,
it always seems as if the military capacity were extinct, this is
because the ancient methods were not good, and no one has arisen
who knew how to discover new ones. Nothing does so much honour to a
newly-risen man than the new laws and measures which he introduces.
These things, when they are well based and have greatness in them,
render him revered and admired, and there is not lacking scope in Italy
for the introduction of every kind. Here there is great virtue in the
members, if it were not wanting in the heads. Look how in duels and in
councils of a few the Italians are superior in strength, dexterity,
and intelligence. But when it comes to armies they make a poor show;
which proceeds entirely from the weakness of the leaders, for those
that know are not obedient, and every one thinks that he knows, there
being hitherto nobody who has raised himself so high both by valour and
fortune as to make the others yield. Hence it comes about that in all
this time, in all the wars waged during the last twenty years, whenever
there has been an army entirely Italian it has always been a failure,
as witness first Taro, then Alexandria, Capua, Genoa, Vaila, Bologna,
and Mestri. If your illustrious house, therefore, wishes to follow
those great men who redeemed their countries, it is before all things
necessary, as the true foundation of every undertaking, to provide
yourself with your own forces, for you cannot have more faithful, or
truer and better soldiers. And although each one of them may be good,
they will together become better when they see themselves commanded
by their prince, and honoured and supported by him. It is therefore
necessary to prepare such forces in order to be able with Italian
prowess to defend the country from foreigners. And although both the
Swiss and Spanish infantry are deemed terrible, none the less they
each have their defects, so that a third order might not only oppose
them, but be confident of overcoming them. For the Spaniards cannot
sustain the attack of cavalry, and the Swiss have to fear infantry
which meets them with resolution equal to their own. From which it has
resulted, as will be seen by experience, that the Spaniards cannot
sustain the attack of French cavalry, and the Swiss are overthrown
by Spanish infantry. And although a complete example of the latter
has not been seen, yet an instance was furnished in the battle of
Ravenna, where the Spanish infantry attacked the German battalions,
which observe the same order as the Swiss. The Spaniards, through their
bodily agility and aided by their bucklers, had entered between and
under their pikes and were in a position to attack them safely without
the Germans being able to defend themselves; and if the cavalry had not
charged them they would have utterly destroyed them. Knowing therefore
the defects of both these kinds of infantry, a third kind can be
created which can resist cavalry and need not fear infantry, and this
will be done not by the creation of armies but by a change of system.
And these are the things which, when newly introduced, give reputation
and grandeur to a new prince. This opportunity must not, therefore,
be allowed to pass, for letting Italy at length see her liberator. I
cannot express the love with which he would be received in all those
provinces which have suffered under these foreign invasions, with what
thirst for vengeance, with what steadfast faith, with what love, with
what grateful tears. What doors would be closed against him? What
people would refuse him obedience? What envy could oppose him? What
Italian would rebel against him? This barbarous domination stinks in
the nostrils of every one. May your illustrious house therefore assume
this task with that courage and those hopes which are inspired by a
just cause, so that under its banner our fatherland may be raised up,
and under its auspices be verified that saying of Petrarch:

\begin{quote}
    Valour against fell wrath\\
    Will take up arms; and be the combat quickly sped I\\
    For, sure, the ancient worth,\\
    That in Italians stirs the heart, is not yet dead.
\end{quote}

\end{document}
